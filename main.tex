\documentclass[a4paper]{article}
\usepackage{student}
\usepackage{graphicx}

\usepackage{cancel}

% Metadata
\date{\today}

%-------------------------------%
% Other details
% TODO: Fill these
%-------------------------------%
\title{Taller 2}
\setmembername{Cristian Camilo Pérez Puentes} 

%-------------------------------%
% Add / Delete commands and packages
% TODO: Add / Delete here as you need
%-------------------------------%
\usepackage{amsmath,amssymb,bm}
\usepackage[spanish]{babel}

% Custom your usual commands here. Renew these.
\newcommand{\KL}{\mathrm{KL}}
\newcommand{\R}{\mathbb{R}}
\newcommand{\E}{\mathbb{E}}
\newcommand{\T}{\top}
\newcommand{\expdist}[2]{%
        \normalfont{\textsc{Exp}}(#1, #2)%
    }
\newcommand{\expparam}{\bm \lambda}
\newcommand{\Expparam}{\bm \Lambda}
\newcommand{\natparam}{\bm \eta}
\newcommand{\Natparam}{\bm H}
\newcommand{\sufstat}{\bm u}

% Main document
\begin{document}
    % Add header
    \header{}

    % Use `answer` environment to add solutions
    % \begin{answer}[Question 1.1] for example starts an environment formatted
    % for Question 1.1
    \begin{align*}
C_V=\left(\frac{\partial U}{\partial T}\right)_V ,\quad C_V=&\left(\frac{\mathrm{d} Q}{d T}\right)_V, \quad B=-V\left(\frac{\partial P}{\partial V}\right)_T \\
C_P=\left(\frac{\mathrm{d} Q}{d T}\right)_P&,\quad C_P=\left(\frac{\mathrm{d} Q}{d T}\right)_P \\
\beta=\frac{1}{V}\left(\frac{\partial V}{\partial T}\right)_P&,\quad \kappa=-\frac{1}{V}\left(\frac{\partial V}{\partial P}\right)_T \\
\left(\frac{\partial x}{\partial y}\right)_z=\frac{1}{(\partial y / \partial x)_z},\quad &\left(\frac{\partial x}{\partial y}\right)_z\left(\frac{\partial y}{\partial z}\right)_x\left(\frac{\partial z}{\partial x}\right)_y=-1 .
\end{align*}
    1.Consideremos que la energía interna de un sistema termodinámico sea una función de T y P,
    obtenga las siguientes ecuaciones:
    \begin{itemize}
        \item  $\quad\left(\frac{\partial U}{\partial T}\right)_P=C_P-P V \beta$.
        \item $\left(\frac{\partial U}{\partial P}\right)_T=P V \kappa-\left(C_P-C_V\right) \frac{\kappa}{\beta}$.
    \end{itemize}
    \begin{answer}[Problema 1]
        \begin{itemize}
            \item[a.]Dado que la energia interna es funcion de T y P, entonces:
            \begin{equation}
                U=U(T,P)
            \end{equation}
            Por lo tanto, la diferencial total de U es:
            \begin{equation}
                dU=\left(\frac{\partial U}{\partial T}\right)_P dT+\left(\frac{\partial U}{\partial P}\right)_T dP
            \end{equation}
            Por otro lado de la primera ley de la termodinamica tenemos que:
            \begin{align*}
                d Q = dU + PdV &=\left( \frac{\partial U}{\partial T} \right)_P dT + \left( \frac{\partial U}{\partial P} \right)_T dP + PdV \quad \Rightarrow\\
                & \Rightarrow \left(
                    \frac{dQ}{dT}
                \right)_P = \left( \frac{\partial U}{\partial T} \right)_P + P \left( \frac{\partial V}{\partial T} \right)_P\\
                & \Rightarrow \left(
                    \frac{dQ}{dT}
                    \right)_P = \left( \frac{\partial U}{\partial T} \right)_P + PV \frac{1}{V} \left( \frac{\partial V}{\partial T} \right)_P\\
                & \Rightarrow \left(
                        C_P = \left( \frac{\partial U}{\partial T} \right)_P + PV \beta
                    \right)\\
                & \Rightarrow \left(
                    \frac{\partial U}{\partial T}
                    \right)_P = C_P - PV \beta
            \end{align*}   
    
            Donde como se realiza $\frac{dQ}{dT}$ a volumen constante entonces 
    
            $$\left(\frac{dV}{dT}\right)_P = \left(\frac{\partial V}{\partial T}\right)_P$$  
            \item [b.] De forma analoga paratiendo de la expresion 
            \begin{align*}
                d Q = dU + PdV &=\left( \frac{\partial U}{\partial T} \right)_P dT + \left( \frac{\partial U}{\partial P} \right)_T dP + PdV \quad \Rightarrow\\
                & \Rightarrow \left(
                    \frac{dQ}{dT}
                \right)_V = \left( \frac{\partial U}{\partial T} \right)_P +  \left( \frac{\partial U}{\partial P} \right)_T \left( \frac{\partial P}{\partial T} \right)_V\\
                & \Rightarrow 
                C_V = C_P - PV \beta - \left( \frac{\partial U}{\partial P} \right)_T \left( \frac{\partial P}{\partial V} \right)_T \left( \frac{\partial V}{\partial T} \right)_P\\
                & \Rightarrow
                    \left( \frac{\partial U}{\partial P} \right)_T = \frac{C_P - PV - C_V}{\left( \frac{\partial P}{\partial V} \right)_T \left( \frac{\partial V}{\partial T} \right)_P}\\
                & \Rightarrow
                    \left( \frac{\partial U}{\partial P} \right)_T = (C_P - PV\beta - C_V)\left( \frac{\partial V}{\partial P} \right)_T \left( \frac{\partial T}{\partial V} \right)_P\\
                & \Rightarrow
                    \left( \frac{\partial U}{\partial P} \right)_T = -(C_P - PV\beta - C_V)\kappa \frac{1}{\beta}\\            \end{align*}\\
        \end{itemize}
        
    \end{answer}
    2. Tomando $\mathrm{U}$ como una función de P y V, obtenga las siguientes ecuaciones:

\begin{align*}
(a)&\quad \mathrm{d} Q=\left(\frac{\partial V}{\partial P}\right)_V d P+\left[\left(\frac{\partial U}{\partial V}\right)_P+P\right] d V.\\
(b)&\quad \left(\frac{\partial U}{\partial P}\right)_V=\frac{C_V \kappa}{\beta}.\\
(c)&\quad \left(\frac{\partial U}{\partial V}\right)_P=\frac{C_P}{V \beta}-P.\\    
\end{align*}
    \begin{answer}
        Si consideramos la energia interna como funcion de $P, V$ es decir $U = U(P,V)$ entonces de la primera ley de la termodinamica para sistema hidrostatico:
    \begin{align*}
        dQ = dU + PdV  \quad (2.1)
    \end{align*}
    \begin{itemize}
        \item [a.]  Como $U = U(P,V)$ entonces:
        \begin{align*}
            dU = \left( \frac{\partial U}{\partial P} \right)_V dP + \left( \frac{\partial U}{\partial V} \right)_P dV \quad (2.2)       
       \end{align*}
       Reemplazando (2.2) en (2.1) tenemos que:
       \begin{align*}
           dQ &= \left( \frac{\partial U}{\partial P} \right)_V dP + \left( \frac{\partial U}{\partial V} \right)_P dV + PdV \\
            &= \left( \frac{\partial U}{\partial P} \right)_V dP + \left[ \left( \frac{\partial U}{\partial V} \right)_P + P \right] dV \quad (2.3)
        \end{align*}     
    \end{itemize} 
    \begin{itemize}
        \item [b.] De (2.3)
        \begin{align*}
            \left(
                \frac{dQ}{dT} 
                \right)_V = \left( \frac{\partial U}{\partial P} \right)_V \left( \frac{dP}{dT}\right)_V \quad  (2.4)
        \end{align*}
        Y como la ecuacion de estado esta en terminos de $P$ y $V$ entonces:
        \begin{align*}
            dT = \left( \frac{\partial T}{\partial P} \right)_V dP + \left( \frac{\partial T}{\partial V} \right)_P dV \quad \Rightarrow \quad \left( \frac{dT}{dP} \right)_V = \left( \frac{\partial T}{\partial P} \right)_V
        \end{align*}
        Luego (2.4) queda:
        \begin{align*}
            \left(
                \frac{dQ}{dT} 
                \right)_V &= \left( \frac{\partial U}{\partial P} \right)_V \left( \frac{\partial T}{\partial P}\right)_V \\
                &= -\left( \frac{\partial U}{\partial P} \right)_V \left( \frac{\partial T}{\partial V}\right)_P \left( \frac{\partial V}{\partial P}\right)_V \\
                &= \left( \frac{\partial U}{\partial P} \right)_V \frac{\beta}{\kappa} = C_V \quad  \Rightarrow
        \end{align*}
        \begin{align*}
            \left( \frac{\partial U}{\partial P} \right)_V = \frac{C_V \kappa}{\beta}
        \end{align*}

        Haciendo uso de  
        \begin{align*}
            C_V = \left( \frac{\partial U}{\partial T} \right)_V  = \left(\frac{dQ}{dT}\right)_V 
        \end{align*}
        \item [c.] De (2.3)

        \begin{align*}
            \left(
                \frac {dQ}{dT}
        \right)_P &= \left( \frac{\partial U}{\partial V} \right)_P \left( \frac{\partial V}{\partial T} \right)_P + P \left( \frac{\partial V}{\partial T} \right)_P\\
        &= \left( \frac{\partial U}{\partial V} \right)_P V \beta + P V \beta = C_P \quad \Rightarrow
        \end{align*}
        \begin{align*}
            \left( \frac{\partial U}{\partial V} \right)_P V\beta = C_P - PV\beta \quad \Rightarrow \quad \left( \frac{\partial U}{\partial V} \right)_P = \frac{C_P}{V\beta} - P
        \end{align*}
        Donde se ha hecho uso de la ecuacion de estado es funcion de $P$ y $V$.
        $$ \left(
            \frac {dT}{dV}
        \right)_P = \left( \frac{\partial T}{\partial V} \right)_P$$
        \end{itemize}
    \end{answer}
    3. Un mol de un gas obedece a la ecuación de estado de van der Waals:
\begin{align*}
\left(P+\frac{a}{v^2}\right)(v-b)=R T    
\end{align*}
Donde a, b y R son constantes, demuestre que
\begin{align*}
    c_p-c_v=\frac{R}{1-2 a(1-b / V)^2 / V R T}
\end{align*}
    \begin{answer}[Punto 3]
    Si consideramos la enegia interna como funcion de $T$ y $V$ es decir $U = U(T,V)$ entonces:
    \begin{align*}
        dU = \left( \frac{\partial U}{\partial T} \right)_V dT + \left( \frac{\partial U}{\partial V} \right)_T dV \quad (3.1)
    \end{align*}
    entonces Reemplazando (3.1) en (2.1) tenemos que:
    \begin{align*}
        &dQ = \left( \frac{\partial U}{\partial T} \right)_V dT + \left( \frac{\partial U}{\partial V} \right)_T dV + PdV \quad (3.2)\quad\\ 
        &\Rightarrow \quad  C_P = \left( \frac{dQ}{dT} \right)_P = \left( \frac{\partial U}{\partial T} \right)_V + \left( \frac{\partial U}{\partial V} \right)_T \left( \frac{\partial V}{\partial T} \right)_P + P \left( \frac{\partial V}{\partial T} \right)_P\\
        &\Rightarrow \quad C_V = \left( \frac {dQ}{dT} \right)_V = \left( \frac{\partial U}{\partial T} \right)_V \quad \Rightarrow
    \end{align*}
    \begin{align*}
        C_P - C_V = \left[ \left( \frac{\partial U}{\partial V} \right)_T + P \right] \left( \frac{\partial V}{\partial T} \right)_P \quad (3.3)
    \end{align*}
    Tambien de (3.2)
    \begin{align*}
        \left(\frac{dQ}{dV}\right)_T = \left( \frac{\partial U}{\partial V} \right)_T + P \quad &\Rightarrow \quad \left( \frac{\partial U}{\partial V} \right)_T = \left(\frac{dQ}{dV}\right)_T - P\\
        &\Rightarrow \quad \left( \frac{\partial U}{\partial V} \right)_T = T \left( \frac{\partial S}{\partial V} \right)_T - P \quad dQ = TdS \quad \text{$T$ constante}\\
        &\Rightarrow \quad \left( \frac{\partial U}{\partial V} \right)_T = T\left( \frac{\partial P}{\partial T} \right)_V - P \quad (3.4) \quad \text{$\left( \frac{\partial S}{\partial V} \right)_T = \left( \frac{\partial P}{\partial T} \right)_V$}
    \end{align*}
    De (3.3) y (3.4)
    \begin{align*}
        C_P - C_V &= \left[ T\left( \frac{\partial P}{\partial T} \right)_V - P + P \right] \left( \frac{\partial V}{\partial T} \right)_P \quad &\Rightarrow \quad C_P - C_V = T\left( \frac{\partial P}{\partial T} \right)_V \left( \frac{\partial V}{\partial T} \right)_P\\
        &=T \left( \frac{\partial P}{\partial T} \right)_V \left( \frac{\partial V}{\partial T} \right)_P \\
     \end{align*}
    
    Del taller anterior sabemos que:
    \begin{align*}
        \left( \frac{\partial P}{\partial T} \right)_V = \frac{R}{V-b} \quad \text{y} \quad \left( \frac{\partial V} {\partial T}\right)_P  = \frac{RV^3}{2ab - aV + PV^3}
    \end{align*}
    Por lo tanto 

    \begin{align*}
        C_P - C_V &= T \frac{R}{V-b} \frac{RV^3}{2ab - aV + PV^3} \\
        &= \frac{R^2TV^3}{2(ab - aV)(V-b) +(V-b)\left(\frac a{V^2} + P\right)V^3}\\
        &= \frac{R^2TV^3}{-2a(V - b)(V-b) +RT V^3}\\
        &= \frac{R}{1 -\frac{2a(V- b)^2}{RTV^3}}\\
    \end{align*}
    \end{answer}
    18. Las ecuaciones de transformación entre dos sistemas de coordenadas son
$$
\begin{gathered}
Q=\log \left(1+q^{1 / 2} \cos p\right) \\
P=2\left(1+q^{1 / 2} \cos p\right) q^{1 / 2} \sin p
\end{gathered}
$$
a) A partir de estas ecuaciones de transformación, demostrar directamente que $Q, P$ son variables canónicas si lo son $q$ y $p$.
b) Demostrar que la función que genera esta transformación es
$$
F_3=-\left(e^Q-1\right)^2 \tan p
$$
    \begin{answer}
        Dado que la ecuacion de estado representa un sistema hidrostatico con $u = u(P,v)$ entonces:

        \begin{align*}
            du = \left( \frac{\partial u}{\partial P} \right)_v dP + \left( \frac{\partial u}{\partial v} \right)_P dv \quad (4.2)
        \end{align*}
        Donde de (4.1):
        \begin{align*}
            \left(\frac {\partial u}{\partial P}\right)_v &= \left(
                \frac{\partial}{\partial P} \left(  P\frac v\Gamma + \frac{f(v)}\Gamma \right)
            \right)_v \\ 
            &= \frac{v}{\Gamma} 
        \end{align*}
        \begin{align*}
            \left(\frac {\partial u}{\partial v}\right)_P &= \left(
                \frac{\partial}{\partial v} \left(  P\frac v\Gamma + \frac{f(v)}\Gamma \right)
            \right)_P\\
            &= \left( \frac{\partial}{\partial v} \left( \frac{f(v)}\Gamma \right) \right)_P + \frac P \Gamma\left( \frac{\partial}{\partial v} \left(  v \right) \right)_P\\
            &= \frac 1\Gamma \left( \frac{\partial f(v)}{\partial v} \right)_P + \frac P \Gamma\\
        \end{align*}
        Reemplazando en (4.2) tenemos que:
        \begin{align*}
            \Gamma dU = v dP + \left( \frac{\partial f(v)}{\partial v} \right)_P dv + P dv \quad  &\Rightarrow \quad \left(\frac {\partial u}{\partial T}\right)_v = \frac v\Gamma \left( \frac{\partial P}{\partial T} \right)_v = c_V\\
            &\Rightarrow \quad c_V =- \frac v\Gamma \left( \frac{\partial P}{\partial v } \right)_T \left( \frac{\partial v}{\partial T} \right)_v\\
            &\Rightarrow \quad c_V = \frac v\Gamma \frac \beta \kappa \\
            &\Rightarrow \quad \Gamma = \frac{\beta v}{c_V \kappa}
        \end{align*}



    \end{answer}


\end{document}