\documentclass[a4paper]{article}
\usepackage{student}
\usepackage{graphicx}
\usepackage{mathrsfs}
\usepackage{cancel}

% Metadata
\date{\today}

%-------------------------------%
% Other details
% TODO: Fill these
%-------------------------------%
\title{Taller 3}
\setmembername{Cristian Perez} 

%-------------------------------%
% Add / Delete commands and packages
% TODO: Add / Delete here as you need
%-------------------------------%
\usepackage{amsmath,amssymb,bm}
\usepackage[spanish]{babel}

% Custom your usual commands here. Renew these.
\newcommand{\KL}{\mathrm{KL}}
\newcommand{\R}{\mathbb{R}}
\newcommand{\E}{\mathbb{E}}
\newcommand{\T}{\top}
\newcommand{\expdist}[2]{%
        \normalfont{\textsc{Exp}}(#1, #2)%
    }
\newcommand{\expparam}{\bm \lambda}
\newcommand{\Expparam}{\bm \Lambda}
\newcommand{\natparam}{\bm \eta}
\newcommand{\Natparam}{\bm H}
\newcommand{\sufstat}{\bm u}

% Main document
\begin{document}
    % Add header
    \header{}

    % Use `answer` environment to add solutions
    % \begin{answer}[Question 1.1] for example starts an environment formatted
    % for Question 1.1
    15. Demostrar directamente que la transformación
$$
\begin{gathered}
Q=\arctan \frac{\alpha q}{p}, \\
P=\frac{\alpha q^2}{2}\left(1+\frac{p^2}{\alpha^2 q^2}\right)
\end{gathered}
$$
es canónica, donde $\alpha$ es una constante.
    \begin{answer}[Problema 1]
       Sean en  $(p,q)$ la coordenadas generalizadas de un sistema en el espacion de fase, y sea $H(p,q)$ la funcion de Hamilton, entonces las ecuaciones de Hamilton son:
        
        \begin{align*}
            \dot p &= -\frac{\partial H}{\partial q}, \quad \dot q = \frac{\partial H}{\partial p} \quad \text{y} \quad  \frac{\partial H}{\partial t} =0  \quad (15.1)
        \end{align*}

        Dado que la transformacion puntual de coordenadas generalizadas $(p,q)$ a coordenadas generalizadas $(P,Q)$ dada por:
        \begin{equation*}
            \begin{align*}  
                Q&=\arctan \frac{\alpha q}{p}, \\
                P&=\frac{\alpha q^2}{2}\left(1+\frac{p^2}{\alpha^2 q^2}\right) = \frac{1}{2}\left(\alpha q^2 + \frac{p^2}{\alpha} \right)
            \end{align*} \quad (15.2)
        \end{equation*}
      
        es canónica, entonces deben satisfacer las ecuaciones de Hamilton (15.1) con $K(P,Q) = K$ constante, es decir:
        \begin{align*}
            \dot P &= -\frac{\partial K}{\partial Q} \quad \text{y} \quad \dot Q = \frac{\partial K}{\partial P}  \quad \text{y} \quad \frac{\partial K}{\partial t} =0  \quad (15.3)
        \end{align*}

        Para mostrar esto ultimo, usando la regla de la cadena en la ecuaciones (15.2) y reemplazando (15.1)
        \begin{align*}
            \frac{dQ}{dt} &= \frac{\partial Q}{\partial q} \frac{dq}{dt} + \frac{\partial Q}{\partial p} \frac{dp}{dt}\\
            & = -\frac{\alpha q}{p^2 + \alpha^2 q^2} \left( -\frac{\partial H}{\partial q} \right) + \frac{\alpha p}{p^2 + \alpha^2 q^2} \left( \frac{\partial H}{\partial p} \right)\\
            &= \frac{\alpha}{p^2 + \alpha^2 q^2} \left( q\frac{\partial H}{\partial q} + p \frac{\partial H}{\partial p} \right)\\
            &= \frac 12\frac{2\alpha }{p^2 + \alpha^2 q^2} \left( q\frac{\partial H}{\partial q} +  p \frac{\partial H}{\partial p} \right)\\
            &= \frac 1 {2P}  \left( q\frac{\partial H}{\partial q} +  p \frac{\partial H}{\partial p} \right) \quad (15.4.1)
        \end{align*}
        \begin{align*}
            \frac{dP}{dt} &= \frac{\partial P}{\partial q} \frac{dq}{dt} + \frac{\partial P}{\partial p} \frac{dp}{dt}\\
            &=  \frac{p}{\alpha} \left( -\frac{\partial H}{\partial q} \right) + q\alpha\left( \frac{\partial H}{\partial p} \right) \quad (15.4.2)
        \end{align*}

        Aplicando la regla de la cadena en (15.1) tenemos que:
        \begin{align*}
            \dot p = -\frac{\partial H}{\partial q} &= -\frac{\partial H}{\partial Q} \frac{\partial Q}{\partial q} - \frac{\partial H}{\partial P} \frac{\partial P}{\partial q}\\
            &= -\frac{\partial H}{\partial Q} \frac{\alpha p}{p^2 + \alpha^2 q^2}  - \frac{\partial H}{\partial P} \alpha q \quad (15.7.1)
        \end{align*}

        \begin{align*}
            \dot q = \frac{\partial H}{\partial p} &= \frac{\partial H}{\partial Q} \frac{\partial Q}{\partial p} + \frac{\partial H}{\partial P} \frac{\partial P}{\partial p}\\
            &= -\frac{\partial H}{\partial Q} \frac{\alpha q}{p^2 + \alpha^2 q^2}  + \frac{\partial H}{\partial P} \frac p\alpha \quad (15.7.2)
        \end{align*}

       

    Reemplazando (15.5) en (15.7.1) y (15.7.2) tenemos que:
    \begin{align*}
        \dot Q &= \frac 1 {2P}  \left( q\frac{\partial H}{\partial q} +  p \frac{\partial H}{\partial p} \right) \\
        &= \frac 1 {2P}  \left( q\left(\frac{\partial H}{\partial Q} \frac{\alpha p}{p^2 + \alpha^2 q^2} + \frac{\partial H}{\partial P} \alpha q\right) +  p \left(-\frac{\partial H}{\partial Q} \frac{\alpha q}{p^2 + \alpha^2 q^2}  + \frac{\partial H}{\partial P} \frac p\alpha \right) \right)\\
        &= \frac 1 {2P}  \left( \frac{\partial H}{\partial Q} \frac{\alpha pq}{p^2 + \alpha^2 q^2}  + \frac{\partial H}{\partial P} \alpha q^2 - \frac{\partial H}{\partial Q} \frac{\alpha pq}{p^2 + \alpha^2 q^2} + \frac{\partial H}{\partial P} \frac{p^2}{\alpha} \right)\\
        &= \frac 1 {2P}  \left( \frac{\partial H}{\partial P} \alpha q^2 + \frac{\partial H}{\partial P} \frac{p^2}{\alpha} \right)\\
        &= \frac 1 {2P}   \frac{\partial H}{\partial P} \left(\alpha q^2 + \frac{p^2}{\alpha} \right) \\
        &= \frac 1 {2P}   \frac{\partial H}{\partial P} 2P = \frac{\partial H}{\partial P}  \\
    \end{align*}

    \begin{align*}
        \dot P &= \frac{p}{\alpha} \left( -\frac{\partial H}{\partial q} \right) + q\alpha\left( \frac{\partial H}{\partial p} \right)\\
        &= \frac{p}{\alpha} \left( -\frac{\partial H}{\partial Q} \frac{\alpha p}{p^2 + \alpha^2 q^2}  - \frac{\partial H}{\partial P} \alpha q\right) + q\alpha\left( -\frac{\partial H}{\partial Q} \frac{\alpha q}{p^2 + \alpha^2 q^2}  + \frac{\partial H}{\partial P} \frac p\alpha \right)\\
        &=  -\frac{\partial H}{\partial Q} \frac{ p^2}{p^2 + \alpha^2 q^2}  - \frac{\partial H}{\partial P} pq  -\frac{\partial H}{\partial Q} \frac{\alpha^2 q^2}{p^2 + \alpha^2 q^2}  + \frac{\partial H}{\partial P} p q\\
        &=  -\frac{\partial H}{\partial Q} \frac{ p^2 + \alpha^2 q^2}{p^2 + \alpha^2 q^2} \\
        &=  -\frac{\partial H}{\partial Q} \\
    \end{align*}

    Por lo que si $K = H(P,Q)$ entonces las anterioes ecuaciones son las ecuaciones de Hamilton (15.3) y por lo tanto la transformacion (15.2) es canónica. Vease que como $H= H(q,p)$ no depende de $t$ entonces $K = H(P,Q)$ tampoco depende de $t$, es decir $\frac{\partial K}{\partial t} = 0$.

    \end{answer}
    2. Tomando $\mathrm{U}$ como una función de P y V, obtenga las siguientes ecuaciones:

\begin{align*}
(a)&\quad \mathrm{d} Q=\left(\frac{\partial V}{\partial P}\right)_V d P+\left[\left(\frac{\partial U}{\partial V}\right)_P+P\right] d V.\\
(b)&\quad \left(\frac{\partial U}{\partial P}\right)_V=\frac{C_V \kappa}{\beta}.\\
(c)&\quad \left(\frac{\partial U}{\partial V}\right)_P=\frac{C_P}{V \beta}-P.\\    
\end{align*}
    \begin{answer}[Punto 2]
        De las ecuaciones de transformacion asociadas a al funcion generatriz $F_2 (q_j, P_j)\quad j =1,\dots n$ dadas por:
        \begin{align*}
            Q_j &= \frac{\partial F_2}{\partial P_j},\quad  p_j = \frac{\partial F_2}{\partial q_j} \quad \text{y} \quad K = H+\frac{\partial F_2}{\partial t}  \quad (16.1)
        \end{align*}
        Se tiene que:
        \begin{align*}
            &\frac{\partial F_2}{\partial t} = 0 \quad \Rightarrow \quad K = H(Q_j,P_j)\\
            &Q_j = \frac{\partial F_2}{\partial P_j} = \frac{\partial}{\partial P_j} (P_j q_j) = q_j\\
            &p_j = \frac{\partial F_2}{\partial q_j} = \frac{\partial}{\partial q_j} (P_j q_j) = P_j\\ 
        \end{align*}
        De esta forma la matriz de transformacion $M$ es tal que:
        \begin{align*}
            M = \begin{pmatrix}
                \frac{\partial Q_i}{\partial q_j} & \frac{\partial Q_i}{\partial p_j}\\
                \frac{\partial P_i}{\partial q_j} & \frac{\partial P_i}{\partial p_j}\\
            \end{pmatrix} =\begin{pmatrix}
                \mathbf 1 & 0\\
                0 & \mathbf 1\\
            \end{pmatrix} =  \begin{pmatrix}
                1 & 0 & \dots & 0\\
                0 & 1 & \dots & 0\\
                \vdots & \vdots & \ddots & \vdots\\
                0 & 0 & \dots & 1\\
            \end{pmatrix}
        \end{align*}
    \end{answer}
    3. Un mol de un gas obedece a la ecuación de estado de van der Waals:
\begin{align*}
\left(P+\frac{a}{v^2}\right)(v-b)=R T    
\end{align*}
Donde a, b y R son constantes, demuestre que
\begin{align*}
    c_p-c_v=\frac{R}{1-2 a(1-b / V)^2 / V R T}
\end{align*}
    \begin{answer}[Punto 3]
        Dado que se busca una funcion generatriz $F'= F'(q_j, p_k,Q_l,P_m,t), \quad j=l,  k=m =1,..,n \text{ o }  j=m,  l=m =1,..,n$, para algun $j,k,l,m$ tal que la transformacion canonica $M$ asociada satisfaga que:
        \begin{align*}
            \mathbf{\dot X} = M \dot {\mathbf{x} }\quad (17.1)
        \end{align*}
        Donode:
        \begin{align*}
            \mathbf X = \begin{pmatrix}
                Q_j\\
                P_j\\
            \end{pmatrix} = \begin{pmatrix}
                p_j\\
                q_j\\
            \end{pmatrix},\quad \mathbf x = \begin{pmatrix}
                q_j\\
                p_j\\
            \end{pmatrix}
            \quad \text{y} \quad 
            M = \begin{pmatrix}
                \frac{\partial Q_j}{\partial q_i} & \frac{\partial Q_j}{\partial p_i}\\
                \frac{\partial P_j}{\partial q_i} & \frac{\partial P_j}{\partial p_i}\\
            \end{pmatrix}
            \quad (17.2)
        \end{align*}
        Por lo que de (17.2):
        \begin{align*}
            M = \begin{pmatrix}
                \frac{\partial Q_j}{\partial q_i} & \frac{\partial Q_j}{\partial p_i}\\
                \frac{\partial P_j}{\partial q_i} & \frac{\partial P_j}{\partial p_i}\\
            \end{pmatrix} = \begin{pmatrix}
                \frac{\partial p_j }{\partial q_i} & \frac{\partial p_j }{\partial p_i}\\
                \frac{\partial q_j }{\partial q_i} & \frac{\partial q_j }{\partial p_i}\\
            \end{pmatrix} = \begin{pmatrix}
                0 & \delta_{ij}\\
                \delta_{ij} & 0\\
            \end{pmatrix} = \begin{pmatrix}
                0 & \mathbf 1\\
                \mathbf 1 & 0\\
            \end{pmatrix} \quad (17.3)
        \end{align*}
        Vease que $M$ de (17.3) no una transformacion canonica, pues:

        \begin{align*}
            M^T J M = \begin{pmatrix}
                0 & \mathbf 1\\
                \mathbf 1 & 0\\
            \end{pmatrix}^T \begin{pmatrix}
                0 & \mathbf 1\\
                -\mathbf 1 & 0\\
            \end{pmatrix} \begin{pmatrix}
                0 & \mathbf 1\\
                \mathbf 1 & 0\\
            \end{pmatrix}
            = \begin{pmatrix}
                0 & \mathbf 1\\
                \mathbf 1 & 0\\
            \end{pmatrix} \begin{pmatrix}
                \mathbf 1 & 0\\
                0 & -\mathbf 1\\
            \end{pmatrix} = \begin{pmatrix}
                0 & -\mathbf 1\\
                \mathbf 1 & 0\\
            \end{pmatrix} \not = J
        \end{align*}
        Por lo que en su lugar se propone:
            \begin{align*}
                \mathbf X = \begin{pmatrix}
                    Q_j\\
                    P_j\\
                \end{pmatrix} = \begin{pmatrix}
                    p_j\\
                    -q_j\\
                \end{pmatrix}
                \quad \text{y} \quad 
                M = \begin{pmatrix}
                    \frac{\partial Q_j}{\partial q_i} & \frac{\partial Q_j}{\partial p_i}\\
                    \frac{\partial P_j}{\partial q_i} & \frac{\partial P_j}{\partial p_i}\\
                \end{pmatrix} \quad (17.4)
            \end{align*}
       

     Asi de (17.4):
        \begin{align*}
            M = \begin{pmatrix}
                \frac{\partial Q_j}{\partial q_i} & \frac{\partial Q_j}{\partial p_i}\\
                \frac{\partial P_j}{\partial q_i} & \frac{\partial P_j}{\partial p_i}\\
            \end{pmatrix} = \begin{pmatrix}
                \frac{\partial p_j }{\partial q_i} & \frac{\partial p_j }{\partial p_i}\\
                \frac{\partial (-q_j) }{\partial q_i} & \frac{\partial (-q_j) }{\partial p_i}\\
            \end{pmatrix} = \begin{pmatrix}
                0 & \delta_{ij}\\
                -\delta_{ij} & 0\\
            \end{pmatrix} = \begin{pmatrix}
                0 & \mathbf 1\\
                -\mathbf 1 & 0\\
            \end{pmatrix} \quad (17.5)
        \end{align*}
        Vease que $M$ de (17.5) es una transformacion canonica, pues:
        \begin{align*}
            M^T J M = \begin{pmatrix}
                0 & \mathbf 1\\
                -\mathbf 1 & 0\\
            \end{pmatrix}^T \begin{pmatrix}
                0 & \mathbf 1\\
                -\mathbf 1 & 0\\
            \end{pmatrix} \begin{pmatrix}
                0 & \mathbf 1\\
                -\mathbf 1 & 0\\
            \end{pmatrix}
            = \begin{pmatrix}
                0 & -\mathbf 1\\
                \mathbf 1 & 0\\
            \end{pmatrix} \begin{pmatrix}
                -\mathbf 1 & 0\\
                0 & -\mathbf 1\\
            \end{pmatrix} = \begin{pmatrix}
                0 & \mathbf 1\\
                -\mathbf 1 & 0\\
            \end{pmatrix} = J
        \end{align*}
        Una una funcion generatriz que me genera este tipo de transformacion canonica  es:
        \begin{align*}
            F(q_j, Q_j,t) = q_jQ_j \quad (17.6)
        \end{align*}
        Pues de (17.6) se tiene que:
        \begin{align*}
            p_i = \frac{\partial F}{\partial q_i} = \frac{\partial}{\partial q_i} (q_jQ_j) = Q_i \quad \text{y} \quad P_i = -\frac{\partial F}{\partial Q_i} = -\frac{\partial}{\partial Q_i} (q_jQ_j) = -q_i
        \end{align*}        
    \end{answer}
    18. Las ecuaciones de transformación entre dos sistemas de coordenadas son
$$
\begin{gathered}
Q=\log \left(1+q^{1 / 2} \cos p\right) \\
P=2\left(1+q^{1 / 2} \cos p\right) q^{1 / 2} \sin p
\end{gathered}
$$
a) A partir de estas ecuaciones de transformación, demostrar directamente que $Q, P$ son variables canónicas si lo son $q$ y $p$.
b) Demostrar que la función que genera esta transformación es
$$
F_3=-\left(e^Q-1\right)^2 \tan p
$$
    \begin{answer}[Punto 4]
        \begin{itemize}
            \item Si $(q,p)$ son variables canónicas entonces satisfacen las ecuaciones canonicas de Hamilton, es decir:
            \begin{align*}
                \dot q &= \frac{\partial H}{\partial p} \quad \text{y} \quad \dot p = -\frac{\partial H}{\partial q} \quad (18.1)
            \end{align*}
            Si ademas define la transformacion:
            \begin{equation*}
                \begin{gathered}
                    Q=\log \left(1+q^{1 / 2} \cos p\right) \\
                    P=2\left(1+q^{1 / 2} \cos p\right) q^{1 / 2} \sin p
                \end{gathered}    \quad (18.2)
            \end{equation*}
            Entonces se tiene que por la regla de la cadena aplicada en (18.1) y (18.2):
            \begin{align*}
                \dot q &= \frac{\partial H}{\partial p} = \frac{\partial H}{\partial P} \frac{\partial P}{\partial p} + \frac{\partial H}{\partial Q} \frac{\partial Q}{\partial p}\\
                &= \frac{\partial H}{\partial P} 2 \left( q^{1/2} \cos p + q \cos (2p) \right) - \frac{\partial H}{\partial Q} \frac{q^{1/2} \sin p}{1 + q^{1/2} \cos p} \\
                &= \frac{\partial H}{\partial P} 2  \left( q^{1/2} \cos p + q \cos (2p) \right) - \frac{\partial H}{\partial Q} \frac{2q \sin^2 p}{P} \quad (18.3.1)
            \end{align*}
            \begin{align*}
                \dot p &= - \frac {\partial H}{\partial q} = -\frac{\partial H}{\partial P} \frac{\partial P}{\partial q} - \frac{\partial H}{\partial Q} \frac{\partial Q}{\partial q}\\
                &= -\frac{\partial H}{\partial P} \left(\frac1{q^{1/2}} + 2\cos p\right)\sin p - \frac{\partial H}{\partial Q} \frac{\cos p}{2(q^{1/2}  + q \cos p) }\\
                &= -\frac{\partial H}{\partial P} \left(\frac1{q^{1/2}} + 2\cos p\right)\sin p - \frac{\partial H}{\partial Q} \frac{\cos p \sin p}{P} \quad (18.3.2)
            \end{align*}
            \begin{align*}
                \dot Q &= \frac{\partial Q}{\partial q} \dot q + \frac{\partial Q}{\partial p} \dot p\\
                &= \frac{\cos p}{2(q^{1/2}  + q \cos p)} \dot q - \frac{q^{1/2} \sin p}{1 + q^{1/2} \cos p} \dot p\\
                &= \frac{\cos p \sin p}{P} \dot q - \frac{2q \sin^2 p}{P} \dot p \quad (18.4.1)
            \end{align*}
            \begin{align*}
                \dot P &= \frac{\partial P}{\partial q} \dot q + \frac{\partial P}{\partial p} \dot p\\
                &= \left(\frac1{q^{1/2}} + 2\cos p\right)\sin p \dot q + 2 \left( q^{1/2} \cos p + q \cos (2p) \right) \dot p \quad (18.4.2) 
            \end{align*}
            Reemplazando (18.3.1) y (18.3.2) en (18.4.1) y (18.4.2):
            \begin{align*}
                \dot Q &= \frac{\cos p \sin p}{P} \left( \frac{\partial H}{\partial P} 2  \left( q^{1/2} \cos p + q \cos (2p) \right) - \frac{\partial H}{\partial Q} \frac{2q \sin^2 p}{P} \right)\\
                &- \frac{2q \sin^2 p}{P} \left( -\frac{\partial H}{\partial P} \left(\frac1{q^{1/2}} + 2\cos p\right)\sin p - \frac{\partial H}{\partial Q} \frac{\cos p \sin p}{P} \right)\\\\
                &= \frac{\cos p \sin p}{P} \left( \frac{\partial H}{\partial P} 2 \left( q^{1/2} \cos p + q \cos^2 p - q \sin^2 p \right) - \frac{\partial H}{\partial Q} \frac{2q \sin^2 p}{P} \right)\\
                &+ \frac{2q \sin^2 p}{P} \left(\frac{\partial H}{\partial P} \left(\frac{1 + q^{1/2}\cos p}{q^{1/2}} + \cos p\right)\sin p + \frac{\partial H}{\partial Q} \frac{\cos p \sin p}{P} \right)\\\\
                &= \frac{\cos p \sin p}{P} \left( \frac{\partial H}{\partial P} 2 \left( q^{1/2}  + q \cos p\right)\frac{\sin p}{\sin p}\cos p - \frac{\partial H}{\partial P} 2\left( q \sin^2 p \right)\right) - \cancel{\frac{\partial H}{\partial Q} \frac{2q \sin^3 p\cos p}{P^2}} \\
                &+ \frac{2q \sin^2 p}{P} \left(\frac{\partial H}{\partial P} 2\left(\frac{q^{1/2} + q\cos p}{2q} \sin p \right) + \frac{\partial H}{\partial P} \cos p\sin p\right) +\cancel{\frac{\partial H}{\partial Q} \frac{2q \sin^3 p\cos p}{P^2}} \\
                &=  \frac{\cos p \sin p}{P} \left( \frac{\partial H}{\partial P}P\frac{\cos p}{\sin p} - \frac{\partial H}{\partial P} 2\left( q \sin^2 p \right)\right)+ \frac{2q \sin^2 p}{P} \left(\frac{\partial H}{\partial P} \left(\frac{P}{2q}  \right) + \frac{\partial H}{\partial P} \cos p\sin p\right)\\
                &=  \frac{\partial H}{\partial P}\cos^2 p - \cancel{2\frac{\partial H}{\partial P}q \sin^3 p \cos p \frac 1 P}+ \frac{\partial H}{\partial P} \sin^2 p + \cancel{2\frac{\partial H}{\partial P} \cos p\sin^3 p \frac 1P}\\
                &=  \frac{\partial H}{\partial P} \quad (18.5.1)\\
            \end{align*}

            \begin{align*}
                \dot P &= \left(\frac1{q^{1/2}} + 2\cos p\right)\sin p \left( \frac{\partial H}{\partial P} 2  \left( q^{1/2} \cos p + q \cos (2p) \right) - \frac{\partial H}{\partial Q} \frac{2q \sin^2 p}{P} \right)\\
                &+ 2 \left( q^{1/2} \cos p + q \cos (2p) \right) \left( -\frac{\partial H}{\partial P} \left(\frac1{q^{1/2}} + 2\cos p\right)\sin p - \frac{\partial H}{\partial Q} \frac{\cos p \sin p}{P} \right)\\\\
                &= \cancel{\left(\frac1{q^{1/2}} + 2\cos p\right)\sin p\frac{\partial H}{\partial P} 2  \left( q^{1/2} \cos p + q \cos (2p) \right)} - \left(\frac1{q^{1/2}} + 2\cos p\right) \frac{\partial H}{\partial Q} \frac{2q \sin^3 p}{P}\\
                &+ \cancel{2 \left( q^{1/2} \cos p + q \cos (2p) \right)\frac{\partial H}{\partial P} \left(\frac1{q^{1/2}} + 2\cos p\right)\sin p} + 2 \left( q^{1/2} \cos p + q \cos (2p) \right) \frac{\partial H}{\partial Q} \frac{\cos p \sin p}{P} \\\\
                &= - \left(\frac{1 + q^{1/2}\cos p}{q^{1/2}} + \cos p\right)\sin p \frac{\partial H}{\partial Q} \frac{2q \sin^3 p}{P} - 2 \left( q^{1/2} \cos p + q \cos^2 p - q \sin^2 p \right) \frac{\partial H}{\partial Q} \frac{\cos p \sin p}{P} \\
            \end{align*}
            \begin{align*}
                &= - \left(\frac {P}{2q} + \cos p\sin p\right) \frac{\partial H}{\partial Q} \frac{2q \sin^2 p}{P} -  \left( 2 \left( q^{1/2}  + q \cos p\right)\frac{\sin p}{\sin p}\cos p -2\left( q \sin^2 p \right)\right) \frac{\partial H}{\partial Q} \frac{\cos p \sin p}{P} \\
                &= - \frac{\partial H}{\partial Q}  \sin^2 - \frac{\partial H}{\partial Q} \frac{2q \sin^3 p \cos p}{P} - \left( P \frac{\cos p}{\sin p} -2\left( q \sin^2 p \right)\right) \frac{\partial H}{\partial Q} \frac{\cos p \sin p}{P} \\
                &= - \frac{\partial H}{\partial Q}  \sin^2 + \cancel{\frac{\partial H}{\partial Q} \frac{2q \sin^3 p \cos p}{P}} - \frac{\partial H}{\partial Q}  \cos^2 p-  \cancel{\frac{\partial H}{\partial Q} \frac{2 q \cos p \sin^3 p}{P} }\\
                &= - \frac{\partial H}{\partial Q} \quad (18.5.2)\\
            \end{align*}
            Por que si $K = H(Q,P)$ entonces las anterioes ecuaciones son las ecuaciones canonicas de Hamilton y por lo tanto $(P,Q)$ son variables canonicas.

            \item  Dado que $F_3 = F_3(p,Q)$, pues:
            \begin{align*}
                F_3 =-\left(e^Q-1\right)^2 \tan p
            \end{align*}
            Entonces de la ecuaciones de canonicas de transformacion para $F_3$:
            \begin{align*}
                q = - \frac{\partial F_3}{\partial p} \quad \text{y} \quad P = \frac{\partial F_3}{\partial Q} \quad (18.6)
            \end{align*}

            se sigue que:
            \begin{align*}
                q = - \frac{\partial F_3}{\partial p} = -\frac{\partial}{\partial p} \left(-\left(e^Q-1\right)^2 \tan p\right) = \left(e^Q-1\right)^2 \sec^2 p \quad (18.7.1)\\
                P = -\frac{\partial F_3}{\partial Q} = -\frac{\partial}{\partial Q} \left(-\left(e^Q-1\right)^2 \tan p\right) = 2\left(e^Q-1\right)e^Q \tan p \quad (18.7.2)
            \end{align*}
            Despejando $Q$ de (18.7.1):
            \begin{align*}
                q = \left(e^Q-1\right)^2 \sec^2 p \quad &\Rightarrow \quad \left(e^Q-1\right)^2 = \frac q{\sec^2 p}\\
                &\Rightarrow \quad e^Q-1 =  \sqrt{\frac q{\sec^2 p}}\\
                &\Rightarrow \quad e^Q = 1 + \sqrt{\frac q{\sec^2 p}}\\
                &\Rightarrow \quad Q = \ln \left(1 + \sqrt{\frac q{\sec^2 p}}\right)\\
                &\Rightarrow \quad Q = \ln \left(1 + q^{1/2} \cos p\right) \quad (18.8.1)
            \end{align*}
            Reemplazando (18.8.1) en (18.7.2):
            \begin{align*}
                P = 2\left(e^Q-1\right)e^Q \tan p \quad &\Rightarrow \quad P = -2\left(1 - q^{1/2} \cos p -1\right)\left(1 - q^{1/2} \cos p\right) \tan p\\
                &\Rightarrow \quad P = -2(1 - 2q^{1/2} \cos p + q \cos^2 p -1 + q^{1/2} \cos p) \tan p\\
                &\Rightarrow \quad P = -2  \left(q^{1/2}\cos p + q\cos^2 p\right) \tan p\\
                &\Rightarrow \quad P = -2q^{1/2} \cos p \left(1 + q^{1/2} \cos p\right) \tan p \quad (18.8.2)
            \end{align*}

            De (18.8.1) y (18.8.2) se concluye que $F_3$ es una funcion generatriz de (18.2) 
        \end{itemize}
    \end{answer}
    5. En el caso de un gas paramagnético, derive la ecuación
$$
dQ=\left(\frac{\partial U}{\partial T}\right)_{V, \mathscr{M}} d T+\left[\left(\frac{\partial U}{\partial V}\right)_{\mathscr{M}, T}+P\right] d V+\left[\left(\frac{\partial U}{\partial \mathscr{M}}\right)_{T, V}-\mu_0 \mathscr{H}\right] d \mathscr{M}
$$
    \begin{answer}[Punto 19]
        Si de las ecuaciones:
        $$
        \begin{array}{cl}
        Q_1=q_1, & P_1=p_1-2 p_2 \\
        Q_2=p_2, & P_2=-2 q_1-q_2
        \end{array} \quad (19.1)
        $$

        Calculamos la matriz de transformación $M$:
        \begin{align*}
            M = \begin{pmatrix}
                \frac{\partial Q_1}{\partial q_1} & \frac{\partial Q_1}{\partial q_2} & \frac{\partial Q_1}{\partial p_1} & \frac{\partial Q_1}{\partial p_2}\\
                \frac{\partial Q_2}{\partial q_1} & \frac{\partial Q_2}{\partial q_2} & \frac{\partial Q_2}{\partial p_1} & \frac{\partial Q_2}{\partial p_2}\\
                \frac{\partial P_1}{\partial q_1} & \frac{\partial P_1}{\partial q_2} & \frac{\partial P_1}{\partial p_1} & \frac{\partial P_1}{\partial p_2}\\
                \frac{\partial P_2}{\partial q_1} & \frac{\partial P_2}{\partial q_2} & \frac{\partial P_2}{\partial p_1} & \frac{\partial P_2}{\partial p_2}\\
            \end{pmatrix} = \begin{pmatrix}
                1 & 0 & 0 & 0\\
                0 & 0 & 0 & 1\\
                0 & 0 & 1 & -2\\
                -2 & -1 & 0 & 0\\
            \end{pmatrix}
        \end{align*}
        Entonces la matriz $M$ es una transformación canónica si y solo si:
        \begin{align*}
            M^T J M = J \quad J = \begin{pmatrix}
                0 & \mathbf 1\\
                -\mathbf 1 & 0\\
            \end{pmatrix} = \begin{pmatrix}
                0 & 0 & 1 & 0\\
                0 & 0 & 0 & 1\\
                -1 & 0 & 0 & 0\\
                0 & -1 & 0 & 0\\
            \end{pmatrix}
        \end{align*}
        Lo cual podemos comprobar:
        \begin{align*}
            M^T J M &= \begin{pmatrix}
                1 & 0 & 0 & 0\\
                0 & 0 & 0 & 1\\
                0 & 0 & 1 & -2\\
                -2 & -1 & 0 & 0\\
            \end{pmatrix}^T \begin{pmatrix}
                0 & 0 & 1 & 0\\
                0 & 0 & 0 & 1\\
                -1 & 0 & 0 & 0\\
                0 & -1 & 0 & 0\\
            \end{pmatrix} \begin{pmatrix}
                1 & 0 & 0 & 0\\
                0 & 0 & 0& 1\\
                0 & 0 & 1 & -2\\
                -2 & -1 & 0 & 0\\
            \end{pmatrix} \\
            &= \begin{pmatrix}
                1 & 0 & 0 & -2\\
                0 & 0 & 0 & -1\\
                0 & 0 & 1 & 0\\
                0 & 1 & -2 & 0\\
            \end{pmatrix} \begin{pmatrix}
                0 & 0 & 1 & -2\\
                -2 & -1 & 0 & 0\\
                -1 & 0 & 0 & 0\\
                0 & 0& 0& -1\\
            \end{pmatrix} \\
            &= \begin{pmatrix}
                0 & 0 & 1 & 0\\
                0 & 0 & 0 & 1\\
                -1 & 0 & 0 & 0\\
                0 & -1& 0 & 0\\
            \end{pmatrix} = J
        \end{align*}
        Por lo tanto $M$ es una transformación canónica.
    
        \item Considere la funcion generatriz $F' = F'(q_1, p_2, P_1, P_2, t)$ tal que:
        \begin{align*}
            F(q_1, q_2, Q_1, Q_2, t)  \rightarrow F''(q_1, p_2, Q_1, Q_2, t) \quad \Rightarrow\\
        \end{align*} 
        \begin{align*}F''(q_1, p_1, Q_1, Q_1, t) &= q_2 \frac{\partial F}{\partial q_2}  - F\\
            &= q_2 p_2 - F
        \end{align*}
        \begin{align*}
            F''(q_1,p_2,Q_1,Q_2,t) \rightarrow F'''(q_1, p_2, P_1, Q_2, t) \quad \Rightarrow 
        \end{align*}
        \begin{align*}
            \quad F'''(q_1,p_2,P_1,Q_2,t) &= Q_1 \frac{\partial F''}{\partial Q_1}  - F''\\
            &= -Q_1 \frac{\partial F}{\partial Q_1}  - q_2 p_2 + F \\
            &= Q_1 P_1 - q_2 p_2 + F(q_1, q_2, Q_1, Q_2, t) \quad (19.2)
        \end{align*}
        \begin{align*}
            F'''(q_1,p_2,P_1,Q_2,t) \rightarrow F'(q_1, p_2, P_1, P_2, t) \quad \Rightarrow 
        \end{align*}
        \begin{align*}
            \quad F'(q_1,p_2,P_1,Q_2,t) &= Q_2 \frac{\partial F'''}{\partial Q_2}  - F'''\\
            &= Q_2 \frac{\partial F}{\partial Q_2}  -Q_1 P_1 +  q_2 p_2 - F \\
            &= -Q_2 P_2 - Q_1 P_1 + q_2 p_2 - F(q_1, q_2, Q_1, Q_2, t) \quad (19.2)
        \end{align*}

        Donde de (19.2) se tiene que:

        \begin{align*}
            \frac{dF}{dt}  &=p_1 \dot q_1 + p_2 \dot q_2 - P_1\dot Q_1- P_2 \dot Q_2 - (H - K)\\
            &= \frac{\partial F}{\partial q_1} \dot q_1 + \frac{\partial F}{\partial q_2} \dot q_2- \frac{\partial F}{\partial Q_1}\dot Q_1- \frac{\partial F}{\partial Q_2} \dot Q_2\\
            &= \frac d{dt}\left(-F'(q_1, p_2, P_1, P_2, t) + q_2 p_2 - Q_1 P_1 - Q_2P_2\right)\\
            &= \frac d{dt}\left(-F'(q_1, p_2, P_1, P_2, t) \right) + \frac d{dt}\left(q_2 p_2 \right) - \frac d{dt}\left(Q_1 P_1 \right) - \frac d{dt}\left(Q_2 P_2 \right)\\
            &= \frac d{dt}\left(-F'(q_1, p_2, P_1, P_2, t) \right) + \dot q_2 p_2 + q_2 \dot p_2 - \dot Q_1 P_1 - Q_1 \dot P_1 - \dot Q_2 P_2 - Q_2 \dot P_2 \quad \Rightarrow\\
        \end{align*}
        
        \begin{align*}
            \frac {dF'}{dt} &= p_1\dot q_1 - q_2 \dot p_2 + Q_1\dot P_1 + Q_2 \dot P_2 - (H-K)\\
            &= \frac{\partial F'}{\partial q_1} \dot q_1 + \frac{\partial F'}{\partial p_2} \dot p_2 + \frac{\partial F'}{\partial P_1}\dot P_1+\frac{\partial F'}{\partial P_2} \dot P_2 + \frac{\partial F'}{\partial t} \quad \Rightarrow\\
        \end{align*}

        \begin{equation*}
            \begin{align*}
                p_1 = \frac{\partial F'}{\partial q_1} & \qquad q_2 = -\frac{\partial F'}{\partial p_2} \\\\
                Q_1 = \frac{\partial F'}{\partial P_1} & \qquad Q_2= \frac{\partial F'}{\partial P_2} \\\\
                \qquad K = H+\frac{\partial F'}{\partial t} 
            \end{align*} \quad (19.3)
        \end{equation*}
        Ahora operando las ecuaciones (19.1)  entonces:
        \begin{align*}
            p_1 = P_1 + 2p_2 &\qquad q_2 = -2q_1 - P_2 \quad (19.4.1)\\
            Q_1 = q_1 &\qquad Q_2 = p_2 \quad (19.4.2)
        \end{align*}
        Comparando (19.3) y (19.4):
        \begin{align*}
                P_1 + 2p_2 = \frac{\partial F'}{\partial q_1} &\qquad  2q_1 + P_2 = \frac{\partial F'}{\partial p_2} \\
                q_1 = \frac{\partial F'}{\partial P_1} &\qquad  p_2= \frac{\partial F'}{\partial P_2} \\
        \end{align*}
        Entonces:
        \begin{align*}
            F'(q_1, p_2, P_1, P_2, t) = q_1 P_1 + p_2 P_2 + 2p_2 q_1 \quad (19.5)
        \end{align*}
        Es la funcion generatriz que genera la transformacion canónica (19.1), pues se puede verificar que:
        \begin{align*}
            p_1 = \frac{\partial F'}{\partial q_1} &= \frac{\partial}{\partial q_1} \left(q_1 P_1 + p_2 P_2 + 2p_2 q_1\right) = -P_1 - 2p_2\\
            q_2 = -\frac{\partial F'}{\partial p_2} &= -\frac{\partial}{\partial p_2} \left(q_1 P_1 + p_2 P_2 + 2p_2 q_1\right) =- P_2 - 2q_1\\
            Q_1 = \frac{\partial F'}{\partial P_1} &= \frac{\partial}{\partial P_1} \left(q_1 P_1 + p_2 P_2 + 2p_2 q_1\right) = q_1\\  
            Q_2 = \frac{\partial F'}{\partial P_2} &= \frac{\partial}{\partial P_2} \left(q_1 P_1 + p_2 P_2 + 2p_2 q_1\right) = p_2\\
        \end{align*}

    \end{answer}
    7. Un mol de un sistema paramagnético ideal obedece la ley de Curie.
$$
\mathscr{M} =\frac{C_{C} \mathscr H}{T}
$$
Donde $\mathscr M$ es la magnetización y $\mathscr H$ es un campo magnético externo, con constante de Curie $C_C$. Suponga que la energía interna $U$ es función de $T$ únicamente, de modo que $\mathrm{dU}=\mathrm{C}_{\mathrm{V},\mathscr M}  \mathrm{dT}$, donde $\mathrm{C}_{\mathrm{v}, \mathscr{M}}$ es una capacidad calorífica a volumen y magnetización constantes. Demostrar que la ecuación de la familia de superficies adiabáticas es
$$
\frac{C_{V, \mathscr M}}{n R} \ln T+\ln V=\frac{\mu_0 \mathscr M^2}{2 n R C_C}+\ln A,
$$
Donde A es una constante.
    \begin{answer}[punto 21]
        De la definicion de bracket de Lagrange:
        \begin{align*}
            \left\{A,B\right\} \equiv \left\{A,B\right\}_{q,p} = \frac{\partial A}{\partial q_i} \frac{\partial B}{\partial p_i} - \frac{\partial  A}{\partial p_i} \frac{\partial B }{\partial q_i} \quad (21.1)
        \end{align*}

        Considerando $u = u(q_i, p_i)$ y $v = v(q_i, p_i)$, y usando la notacion matricial:
        \begin{align*}
            \frac{\partial u}{\partial \pmb \eta} = \frac {\partial u}{\partial (q_i, p_i)} = \begin{pmatrix}
                \frac{\partial u}{\partial q_i} \\
                \frac{\partial u}{\partial p_i} \\
            \end{pmatrix} \quad \text{y} \quad \frac{\partial v}{\partial \pmb \eta} = \frac {\partial v}{\partial (q_i, p_i)} = \begin{pmatrix}
                \frac{\partial v}{\partial q_i} \\
                \frac{\partial v}{\partial p_i} \\
            \end{pmatrix}
        \end{align*}
        De esta forma
        \begin{align*}
            \left[u,v\right] &= \tilde{\frac{\partial u}{\partial \pmb \eta}} J {\frac{\partial v}{\partial \pmb \eta}}\\
            &= \begin{pmatrix}
                \frac{\partial u}{\partial q_i} &
                \frac{\partial u}{\partial p_i} 
            \end{pmatrix} 
            \begin{pmatrix}
                0 & \mathbf 1\\
                -\mathbf 1 & 0\\
            \end{pmatrix}
            \begin{pmatrix}
                \frac{\partial v}{\partial q_i} \\
                \frac{\partial v}{\partial p_i} \\
            \end{pmatrix}\\
            &= \begin{pmatrix}
                \frac{\partial u}{\partial q_i} &
                \frac{\partial u}{\partial p_i}
            \end{pmatrix}
            \begin{pmatrix}
                \frac{\partial v}{\partial p_i} \\
                -\frac{\partial v}{\partial q_i} \\
            \end{pmatrix}\\
            &= \frac{\partial u}{\partial q_i} \frac{\partial v}{\partial p_i} - \frac{\partial u}{\partial p_i} \frac{\partial v}{\partial q_i}\\
        \end{align*}
        En notacion tensorial $\partial_i u \equiv \frac{\partial u}{\partial \eta_i}$ y $\partial_i v \equiv \frac{\partial v}{\partial \eta_i}$ entonces:
        \begin{align*}
            \left[u,v\right] &= \partial_i u J_{ij} \partial_j v\\
        \end{align*}

        Por lo que para $A=A(q_i, p_i)$ y $B=B(q_i, p_i)$ y $C=C(q_i, p_i)$  funciones analiticas, se tiene que:
        \begin{align*}
            \left\{A, \left\{B,C \right\} \right\} &= \left\{A, \partial_i B J_{ij} \partial_j C \right\}\\
            &= \partial_i A J_{ij} \partial_j \left(\partial_k B J_{kl} \partial_l C \right)\\
            &= \partial_i A J_{ij} \left(\partial_j \partial_k B J_{kl} \partial_l C + \partial_k B J_{kl} \partial_j \partial_l C \right)\\
            &=  \partial_i A J_{ij}\partial_j \partial_k B J_{kl} \partial_l C + \partial_i A J_{ij}\partial_k B J_{kl} \partial_j \partial_l C \\
            &= \partial_l C  J_{kl} \partial_i A J_{ij}\partial_k \partial_j B + \partial_k B J_{kl}\partial_j \partial_l C \partial_i A J_{ij} \\
            &= -\partial_l C  J_{lk} \partial_k \partial_j B \partial_i A J_{ij} + \partial_k B J_{kl} \partial_i A J_{ij} \partial_j \partial_l C   \\
            &= -\partial_l C  J_{lk}\partial_k \partial_j B J_{ij}\partial_i A + \partial_k B J_{kl} \partial_i A J_{ij} \partial_j \partial_l C  \\
            &= \partial_l C  J_{lk} \partial_k \partial_j BJ_{ji}\partial_i A + \partial_l C  J_{lk} \partial_j BJ_{ji}\partial_k  \partial_i A_i + \partial_l C  J_{lk} \partial_j BJ_{ij}\partial_k  \partial_i A_i + \partial_k B J_{kl} \partial_i A J_{ij} \partial_l \partial_j C \\
            &= \partial_l C  J_{lk}\left(\partial_k \partial_j BJ_{ji}\partial_i A + \partial_j BJ_{ji}\partial_k  \partial_i A\right) + \partial_j B J_{ji}\partial_l\partial_i A  J_{lk}   \partial_k C   + \partial_k B J_{kl} \partial_i A J_{ij} \partial_l \partial_j C\\
            &= \partial_l C  J_{lk}\partial_k\left(\partial_j BJ_{ji} \partial_i A\right) + \partial_k B J_{kl}\partial_i\partial_l A  J_{ij}   \partial_j C   + \partial_k B J_{kl} \partial_i A J_{ij} \partial_l \partial_j C\\
            &= -\partial_l C  J_{lk}\partial_k\left(\partial_i A J_{ij} \partial_j B\right) + \partial_k B J_{kl}\partial_l(\partial_i A  J_{ij}   \partial_j C)  \\
            &= -\partial_l C  J_{lk}\partial_k\left\{A,B\right\} - \partial_k B J_{kl}\partial_l\left\{C,A\right\}  \\
            &= -\left\{C,\left\{A,B\right\}\right\} - \left\{B,\left\{C,A\right\}\right\}  \quad \Rightarrow  \\
        \end{align*}
        \begin{align*}
            \left\{A, \left\{B,C \right\} \right\} + \left\{B, \left\{C,A \right\} \right\} + \left\{C, \left\{A,B \right\} \right\} = 0 \quad (21.2)
        \end{align*}



        \begin{align*}
            J_{ij} J_{kl}
        \end{align*}

      
     


    \end{answer}
    8. La figura 1 , se representa un diagrama $PV$ simplificado del ciclo de gas ideal de Joule. Todos los procesos son cuasi-estáticos y $\mathrm{C_P}$ es constante. Demuestre que la eficiencia térmica de un motor que realiza este ciclo es
$$
\eta=1-\left(\frac{P_{\mathbf{1}}}{P_2}\right)^{(\gamma-1) / \gamma}
$$
figura 1. Ciclo de gas ideal Joule

\begin{figure}[h]
    \centering
    \includegraphics[width=0.5\textwidth]{diagrama.png}
    \label{fig:figura1}    
\end{figure}
    \begin{answer}[Punto 22]
        Partiendo de las ecuaciones canonicas de Hamilton:
        \begin{align*}
            \dot q_i &= \frac{\partial H}{\partial p_i} \quad \text{y} \quad \dot p_i = -\frac{\partial H}{\partial q_i} \quad  \frac{\partial H }{\partial t} = -\frac{\partial L}{\partial t} \quad (22.1)
        \end{align*}
        Y sea $u = u(q_i, p_i, t)$, entonces:
        \begin{align*}
            \frac{du}{dt} &= \frac{\partial u}{\partial q_i} \dot q_i + \frac{\partial u}{\partial p_i} \dot p_i + \frac{\partial u}{\partial t}\\
            &= \frac{\partial u}{\partial q_i} \frac{\partial H}{\partial p_i} - \frac{\partial u}{\partial p_i} \frac{\partial H}{\partial q_i} + \frac{\partial u}{\partial t}\\
            &= \left\{u, H\right\} + \frac{\partial u}{\partial t} \quad (22.2)
        \end{align*}

        Si definimos: $\pmb \eta \equiv (q_i(t), p_i(t))$, tal que $\dot {\pmb \eta} = (\dot q_i, \dot p_i)$, entonces:
        \begin{align*}
            \dot {\pmb \eta} &= \begin{pmatrix}
                \dot q_i \\
                \dot p_i \\
            \end{pmatrix} = \begin{pmatrix}
                \frac{\partial H}{\partial p_i} \\
                -\frac{\partial H}{\partial q_i} \\
            \end{pmatrix}\\
             &= \begin{pmatrix}
                0 & \mathbf 1\\
                -\mathbf 1 & 0\\
            \end{pmatrix} \begin{pmatrix}
                \frac{\partial H}{\partial q_i} \\
                \frac{\partial H}{\partial p_i} \\
            \end{pmatrix} 
            = J \frac{\partial H}{\partial \pmb \eta} = \tilde{\frac{\partial \pmb \eta}{\partial \pmb \eta}} J \frac{\partial H}{\partial \pmb \eta} \\
            &= \left\{\pmb \eta, H\right\} \quad  \Rightarrow 
        \end{align*}

        \begin{align*}
            \dot {\pmb \eta} = \left\{\pmb \eta, H\right\} \quad (22.2)
        \end{align*}
        Son las ecuaciones canonicas de Hamilton en corchetes de Poisson, con $\frac{\partial \pmb \eta}{\partial t} =0$, pues podemos notar que:
        \begin{align*}
            \left\{q_i, H \right\} = \tilde{\frac{\partial  q_i}{\partial \pmb \eta}} J \frac{\partial H}{\partial \pmb \eta} = \begin{pmatrix} 1 \\ 0 \end{pmatrix} \begin{pmatrix} 0 & 1 \\ -1 & 0 \end{pmatrix} \begin{pmatrix} \frac{\partial H}{\partial q_i} \\ \frac{\partial H}{\partial p_i} \end{pmatrix} = \frac{\partial H}{\partial p_i} = \dot q_i\\
        \end{align*}
        \begin{align*}
            \left\{p_i, H \right\} = \tilde{\frac{\partial  p_i}{\partial \pmb \eta}} J \frac{\partial H}{\partial \pmb \eta} = \begin{pmatrix} 0 \\ 1 \end{pmatrix} \begin{pmatrix} 0 & 1 \\ -1 & 0 \end{pmatrix} \begin{pmatrix} \frac{\partial H}{\partial q_i} \\ \frac{\partial H}{\partial p_i} \end{pmatrix} = -\frac{\partial H}{\partial q_i} = \dot p_i\\
        \end{align*}
        \begin{align*}
            \left\{H, H \right\} + \frac{\partial H}{\partial t}= \tilde{\frac{\partial  H}{\partial \pmb \eta}} J \frac{\partial H}{\partial \pmb \eta} = \begin{pmatrix} \frac{\partial H}{\partial q_i} \\ \frac{\partial H}{\partial p_i} \end{pmatrix} \begin{pmatrix} 0 & 1 \\ -1 & 0 \end{pmatrix} \begin{pmatrix} \frac{\partial H}{\partial q_i} \\ \frac{\partial H}{\partial p_i} \end{pmatrix} = \frac{d H}{d t}
        \end{align*}
    \end{answer}
    9. (a) Deduzca la expresión para la eficiencia de un motor de Carnot directamente de un diagrama TS (Temperatura vs Entropía). (b) Compare las eficiencias de los ciclos A y B de la Figura 2.
figura 2.

\begin{figure}[h]
    \centering
    \includegraphics[width=0.5\textwidth]{diagram2.png}
    
    \label{fig:figura2}
\end{figure}

    \begin{answer}[Punto 23]
    Sean $q_{ji}$ y $p_{ji}$ coordenadas represntando las posiciones y momentos usuales, tales que $p_{ji} = m_i \dot q_{ji} $, con $i = 1,2,3$ entonces:
    \begin{align*}
        \left\{\mathbf{p}, \mathbf{L}\right\} &=  \tilde{\frac{\partial \mathbf p}{\partial \pmb \eta}}  { J {\frac{\partial \mathbf L}{\partial \pmb \eta}} }=\tilde{\frac{\mathbf p}{\partial \pmb \eta}}J  {{\frac{\partial}{\partial \pmb \eta}}}\left(\mathbf x \times \mathbf p\right)  \quad (23.1)\\
    \end{align*}
    donde 

    \begin{align*}
        \frac{\partial \mathbf C}{\partial \pmb \eta} &= \begin{pmatrix}
            \frac{\partial  C_x}{\partial q_{i}}  &  \frac{\partial  C_y}{\partial q_{i}}  &  \frac{\partial  C_z}{\partial q_{i}}  \\
              \frac{\partial  C_x}{\partial p_{i}}  &  \frac{\partial  C_y}{\partial p_{i}}  &  \frac{\partial  C_z}{\partial p_{i}}  \\
        \end{pmatrix} =
        \begin{pmatrix}
            \frac{\partial  (A_y B_z - B_y A_z)}{\partial q_{i}}  &  \frac{\partial  (A_z B_x - B_x A_z)}{\partial q_{i}}  &  \frac{\partial  (A_x B_y - A_y B_x)}{\partial q_{i}}  \\
                \frac{\partial  (A_y B_z - B_y A_z)}{\partial p_{i}}  &  \frac{\partial  (A_z B_x - B_x A_z)}{\partial p_{i}}  &  \frac{\partial  (A_x B_y - A_y B_x)}{\partial p_{i}}  \\
        \end{pmatrix}\\\\
        &= \begin{pmatrix}
            \frac{\partial  A_y B_z}{\partial q_{i}}  - \frac{\partial  A_z B_y}{\partial q_{i}}  &  \frac{\partial  A_z B_x}{\partial q_{i}}  - \frac{\partial  B_x A_z}{\partial q_{i}}  &  \frac{\partial  A_x B_y}{\partial q_{i}}  - \frac{\partial  A_y B_x}{\partial q_{i}}  \\
                \frac{\partial  A_y B_z}{\partial p_{i}}  - \frac{\partial  A_z B_y}{\partial p_{i}}  &  \frac{\partial  A_z B_x}{\partial p_{i}}  - \frac{\partial  B_x A_z}{\partial p_{i}}  &  \frac{\partial  A_x B_y}{\partial p_{i}}  - \frac{\partial  A_y B_x}{\partial p_{i}}  \\
        \end{pmatrix}\\
        &= \begin{pmatrix}
            A_y \frac{\partial  B_z}{\partial q_{i}}  + B_z \frac{\partial  A_y}{\partial q_{i}}  &  A_z \frac{\partial  B_x}{\partial q_{i}}  + B_x \frac{\partial  A_z}{\partial q_{i}}  &  A_x \frac{\partial  B_y}{\partial q_{i}}  + B_y \frac{\partial  A_x}{\partial q_{i}}  \\
              A_y \frac{\partial  B_z}{\partial p_{i}}  + B_z \frac{\partial  A_y}{\partial p_{i}}  &  A_z \frac{\partial  B_x}{\partial p_{i}}  + B_x \frac{\partial  A_z}{\partial p_{i}}  &  A_x \frac{\partial  B_y}{\partial p_{i}}  + B_y \frac{\partial  A_x}{\partial p_{i}}  \\
        \end{pmatrix} \\
        &- \begin{pmatrix}
            A_z \frac{\partial  B_y}{\partial q_{i}}  + B_y \frac{\partial  A_z}{\partial q_{i}}  &  A_x \frac{\partial  B_z}{\partial q_{i}}  + B_z \frac{\partial  A_x}{\partial q_{i}}  &  A_y \frac{\partial  B_x}{\partial q_{i}}  + B_x \frac{\partial  A_y}{\partial q_{i}}  \\
              A_z \frac{\partial  B_y}{\partial p_{i}}  + B_y \frac{\partial  A_z}{\partial p_{i}}  &  A_x \frac{\partial  B_z}{\partial p_{i}}  + B_z \frac{\partial  A_x}{\partial p_{i}}  &  A_y \frac{\partial  B_x}{\partial p_{i}}  + B_x \frac{\partial  A_y}{\partial p_{i}}  \\
        \end{pmatrix} \\\\
        & =\begin{pmatrix}
            A_y \frac{\partial  B_z}{\partial q_{i}}  & A_z \frac{\partial  B_x}{\partial q_{i}}  &  A_x \frac{\partial  B_y}{\partial q_{i}}  \\
              A_y \frac{\partial  B_z}{\partial p_{i}}  &  A_z \frac{\partial  B_x}{\partial p_{i}}  &  A_x \frac{\partial  B_y}{\partial p_{i}}  \\
        \end{pmatrix} + \begin{pmatrix}
            B_z \frac{\partial  A_y}{\partial q_{i}}   &  B_x \frac{\partial  A_z}{\partial q_{i}}   &  B_y \frac{\partial  A_x}{\partial q_{i}}   \\
              B_z \frac{\partial  A_y}{\partial p_{i}}   &  B_x \frac{\partial  A_z}{\partial p_{i}}   &  B_y \frac{\partial  A_x}{\partial p_{i}}   \\
        \end{pmatrix}\\
        -&\begin{pmatrix}
            A_z \frac{\partial  B_y}{\partial q_{i}}   &  A_x \frac{\partial  B_z}{\partial q_{i}}   &  A_y \frac{\partial  B_x}{\partial q_{i}}   \\
              A_z \frac{\partial  B_y}{\partial p_{i}}   &  A_x \frac{\partial  B_z}{\partial p_{i}}   &  A_y \frac{\partial  B_x}{\partial p_{i}}   \\
        \end{pmatrix} - \begin{pmatrix}
            B_y \frac{\partial  A_z}{\partial q_{i}}   &  B_z \frac{\partial  A_x}{\partial q_{i}}   &  B_x \frac{\partial  A_y}{\partial q_{i}}   \\
              B_y \frac{\partial  A_z}{\partial p_{i}}   &  B_z \frac{\partial  A_x}{\partial p_{i}}   &  B_x \frac{\partial  A_y}{\partial p_{i}}   \\
        \end{pmatrix}\\\\
        & = \frac{\partial \mathbf B}{\partial \pmb \eta} \begin{pmatrix}
            0 & 1 & 0\\
            0 & 0 & 1\\
            1 & 0 & 0
        \end{pmatrix}  \begin{pmatrix}
            A_y & 0 & 0\\
            0 & A_z & 0\\
            0 & 0 & A_x
        \end{pmatrix}   +  \frac{\partial \mathbf A}{\partial \pmb \eta}\begin{pmatrix}
            0 & 0 & 1\\
            1 & 0 & 0\\
            0 & 1 & 0
        \end{pmatrix} \begin{pmatrix}
            B_z & 0 & 0\\
            0 & B_x & 0\\
            0 & 0 & B_y
        \end{pmatrix}\\
        &- \frac{\partial \mathbf B}{\partial \pmb \eta}\begin{pmatrix}
            0 & 0 & 1\\
            1 & 0 & 0\\
            0 & 1 & 0
        \end{pmatrix} \begin{pmatrix}
            A_z& 0 & 0\\
            0 & A_x & 0\\
            0 & 0 & A_y
        \end{pmatrix} -  \frac{\partial \mathbf A}{\partial \pmb \eta}\begin{pmatrix}
            0 & 1 & 0\\
            0 & 0 & 1\\
            1 & 0 & 0
        \end{pmatrix}  \begin{pmatrix}
            B_y & 0 & 0\\
            0 & B_z & 0\\
            0 & 0 & B_x
        \end{pmatrix}\\\\
        & = \frac{\partial \mathbf B}{\partial \pmb \eta} \begin{pmatrix}
            0 & A_z & 0\\
            0 & 0 & A_x\\
            A_y& 0 & 0
        \end{pmatrix}    +  \frac{\partial \mathbf A}{\partial \pmb \eta}\begin{pmatrix}
            0 & 0 & B_y\\
            B_z & 0 & 0\\
            0 & B_x & 0
        \end{pmatrix}\\
        &- \frac{\partial \mathbf B}{\partial \pmb \eta}\begin{pmatrix}
            0 & 0 & A_y\\
            A_z & 0 & 0\\
            0 & A_x & 0
        \end{pmatrix} -  \frac{\partial \mathbf A}{\partial \pmb \eta}\begin{pmatrix}
            0 & B_z & 0\\
            0 & 0 & B_x\\
            B_y & 0 & 0
        \end{pmatrix}\\\\
        & = \frac{\partial \mathbf B}{\partial \pmb \eta} \begin{pmatrix}
            0 & A_z & -A_y\\
            -A_z & 0 & A_x\\
            A_y& -A_x & 0
        \end{pmatrix}    +  \frac{\partial \mathbf A}{\partial \pmb \eta}\begin{pmatrix}
            0 & -B_z & B_y\\
            B_z & 0 & -B_x\\
            -B_y & B_x & 0
        \end{pmatrix}\\
        \end{align*}
        En terminos del algebra de Dyadicas se pude escribir:
        \begin{align*}
                \frac{\partial \mathbf C}{\partial \pmb \eta} &= \frac{\partial}{\partial \pmb \eta}(\mathbf A \times \mathbf B)\\
                &= \frac{\partial \mathbf B}{\partial \pmb \eta} \left(
                    A_z \mathbf {\hat e}_1 \mathbf {\hat e}_2 - A_y \mathbf {\hat e}_1 \mathbf {\hat e}_3 + A_x \mathbf {\hat e}_2 \mathbf {\hat e}_3 
                    - A_z \mathbf {\hat e}_2 \mathbf {\hat e}_1 + A_y \mathbf {\hat e}_3 \mathbf {\hat e}_1 - A_x \mathbf {\hat e}_3 \mathbf {\hat e}_2
                \right) \\
                &+ \frac{\partial \mathbf A}{\partial \pmb \eta} \left(
                    -B_z \mathbf {\hat e}_1 \mathbf {\hat e}_2 + B_y \mathbf {\hat e}_1 \mathbf {\hat e}_3 - B_x \mathbf {\hat e}_2 \mathbf {\hat e}_3
                    + B_z \mathbf {\hat e}_2 \mathbf {\hat e}_1 - B_y \mathbf {\hat e}_3 \mathbf {\hat e}_1 + B_x \mathbf {\hat e}_3 \mathbf {\hat e}_2
                \right)\\
                &= \frac{\partial \mathbf B}{\partial \pmb \eta} \left(
                    A_z\left(\mathbf {\hat e}_1 \mathbf {\hat e}_2 - \mathbf {\hat e}_2 \mathbf {\hat e}_1\right) + A_y\left(\mathbf {\hat e}_3 \mathbf {\hat e}_1 - \mathbf {\hat e}_1 \mathbf {\hat e}_3\right) + A_x\left(\mathbf {\hat e}_2 \mathbf {\hat e}_3 - \mathbf {\hat e}_3 \mathbf {\hat e}_2\right)
                \right)\\
                &+ \frac{\partial \mathbf A}{\partial \pmb \eta} \left(
                    -B_z\left(\mathbf {\hat e}_1 \mathbf {\hat e}_2 - \mathbf {\hat e}_2 \mathbf {\hat e}_1\right) - B_y\left(\mathbf {\hat e}_3 \mathbf {\hat e}_1 - \mathbf {\hat e}_1 \mathbf {\hat e}_3\right) - B_x\left(\mathbf {\hat e}_2 \mathbf {\hat e}_3 - \mathbf {\hat e}_3 \mathbf {\hat e}_2\right)
                \right)\\
                &= \frac{\partial \mathbf B}{\partial \pmb \eta} \left(
                    A_i \mathbf{\hat e}_j \mathbf{\hat e}_k \epsilon_{ijk}
                \right) - \frac{\partial \mathbf A}{\partial \pmb \eta} \left(
                    B_i \mathbf{\hat e}_j \mathbf{\hat e}_k \epsilon_{ijk}
                \right)\\
                &= \frac{\partial \mathbf B}{\partial \pmb \eta} \left(
                    A_i \mathbf{\hat e}_i \cdot \left(\mathbf {\hat e}_j \mathbf{\hat e}_k \mathbf{\hat e}_l\right) \epsilon_{jkl}
                \right) - \frac{\partial \mathbf A}{\partial \pmb \eta} \left(
                    B_i \mathbf{\hat e}_i \cdot \left(\mathbf {\hat e}_j \mathbf{\hat e}_k \mathbf{\hat e}_l\right) \epsilon_{jkl}
                \right)\\
                &= \frac{\partial \mathbf B}{\partial \pmb \eta} \left(
                    \mathbf A \cdot \begin{vmatrix}
                        \mathbf {\hat e}_1 & \mathbf{\hat e}_2 & \mathbf{\hat e}_3\\
                        \mathbf {\hat e}_1 & \mathbf{\hat e}_2 & \mathbf{\hat e}_3\\
                        \mathbf {\hat e}_1 & \mathbf{\hat e}_2 & \mathbf{\hat e}_3\\
                    \end{vmatrix}
                \right) - \frac{\partial \mathbf A}{\partial \pmb \eta} \left(
                    \mathbf B \cdot \begin{vmatrix}
                        \mathbf {\hat e}_1 & \mathbf{\hat e}_2 & \mathbf{\hat e}_3\\
                        \mathbf {\hat e}_1 & \mathbf{\hat e}_2 & \mathbf{\hat e}_3\\
                        \mathbf {\hat e}_1 & \mathbf{\hat e}_2 & \mathbf{\hat e}_3\\
                    \end{vmatrix}
                \right)\\
                &= \frac{\partial \mathbf B}{\partial \pmb \eta} \left( \mathbf A \cdot \bar{\bar{\pmb \epsilon}} 
                \right) - \frac{\partial \mathbf A}{\partial \pmb \eta} \left( \mathbf B\cdot \bar{\bar{\pmb \epsilon}} 
                \right) \quad \Rightarrow \\
        \end{align*}
        \begin{align*}
            \frac{\partial } {\partial \mathbf{x}} \left(\mathbf A \times \mathbf B\right) = \frac{\partial \mathbf B}{\partial \mathbf x} \left( \mathbf A \cdot \bar{\bar{\pmb \epsilon}}
            \right) - \frac{\partial \mathbf A}{\partial \mathbf{x}} \left( \mathbf B\cdot \bar{\bar{\pmb \epsilon}}
            \right)
        \end{align*}
        Es claro que 
        \begin{align*}
            \bar{\bar {\pmb \epsilon}} = \begin{vmatrix}
                \mathbf {\hat e}_1 & \mathbf{\hat e}_2 & \mathbf{\hat e}_3\\
                \mathbf {\hat e}_1 & \mathbf{\hat e}_2 & \mathbf{\hat e}_3\\
                \mathbf {\hat e}_1 & \mathbf{\hat e}_2 & \mathbf{\hat e}_3\\
            \end{vmatrix} = \mathbf {\hat e}_i \mathbf{\hat e}_j \mathbf{\hat e}_k \epsilon_{ijk}
        \end{align*}
        Es el pseudo tensor de rango 3 de Levi-Civita. Ademas:
        \begin{align*}
            \tilde{\frac{\partial}{\partial \pmb \eta}}\left(\mathbf A \times \mathbf B\right) &= \left[ \frac{\partial \mathbf B}{\partial \pmb \eta} \left( \mathbf A \cdot \bar{\bar{\pmb \epsilon}} 
            \right) - \frac{\partial \mathbf A}{\partial \pmb \eta} \left( \mathbf B\cdot \bar{\bar{\pmb \epsilon}} 
            \right)\right]^T \\
            &= \left[ \frac{\partial \mathbf B}{\partial \pmb \eta} \left( \mathbf A \cdot \bar{\bar{\pmb \epsilon}}
            \right)\right]^T - \left[\frac{\partial \mathbf A}{\partial \pmb \eta} \left( \mathbf B\cdot \bar{\bar{\pmb \epsilon}}
            \right)\right]^T\\
            &= \left( \mathbf A \cdot \bar{\bar{\pmb \epsilon}}\right)^T \left(\frac{\partial \mathbf B}{\partial \pmb \eta}\right)^T - \left( \mathbf B\cdot \bar{\bar{\pmb \epsilon}}\right)^T \left(\frac{\partial \mathbf A}{\partial \pmb \eta}\right)^T\\
            &= \left( \mathbf B\cdot \bar{\bar{\pmb \epsilon}}\right) \tilde{\frac{\partial \mathbf A}{\partial \pmb \eta}}-\left( \mathbf A \cdot \bar{\bar{\pmb \epsilon}}\right) \tilde{\frac{\partial \mathbf B}{\partial \pmb \eta}} \quad (23.2)
        \end{align*}
        De este modo (23.1) se puede escribir como:
        \begin{align*}
            \left\{\mathbf{p}, \mathbf{L}\right\} &=  \tilde{\frac{\partial \mathbf p}{\partial \pmb \eta}} J  {\frac{\partial \mathbf L}{\partial \pmb \eta}} = \tilde{\frac{\partial \mathbf p}{\partial \pmb \eta}} J  {\frac{\partial}{\partial \pmb \eta}}\left(\mathbf q \times \mathbf p\right) \\
            &= \tilde{\frac{\partial \mathbf p}{\partial \pmb \eta}} J \left[\frac{\partial \mathbf p}{\partial \pmb \eta} \left( \mathbf q \cdot \bar{\bar{\pmb \epsilon}} 
            \right) - \frac{\partial \mathbf q}{\partial \pmb \eta} \left( \mathbf p\cdot \bar{\bar{\pmb \epsilon}} 
            \right)\right]\\
            &= \tilde{\frac{\partial \mathbf p}{\partial \pmb \eta}} J \frac{\partial \mathbf p}{\partial \pmb \eta} \left( \mathbf q \cdot \bar{\bar{\pmb \epsilon}} 
            \right) - \tilde{\frac{\partial \mathbf p}{\partial \pmb \eta}} J \frac{\partial \mathbf q}{\partial \pmb \eta} \left( \mathbf p\cdot \bar{\bar{\pmb \epsilon}}
            \right)\\
            &= \cancelto{0}{\left\{ \mathbf p, \mathbf p \right\}} \left( \mathbf q \cdot \bar{\bar{\pmb \epsilon}} 
            \right) - \left\{ \mathbf p, \mathbf q \right\} \left( \mathbf p\cdot \bar{\bar{\pmb \epsilon}}
            \right)\\
            &=  \cancelto{\mathbb{I}}{\left\{ \mathbf q, \mathbf p \right\}} \left( \mathbf p \cdot \bar{\bar{\pmb \epsilon}}\right) =  \mathbf p \cdot \bar{\bar{\pmb \epsilon}}\\
        \end{align*}
        Y de forma analoga
        \begin{align*}
            \{\mathbf q, \mathbf L\} &= \tilde{\frac{\partial \mathbf q}{\partial \pmb \eta}} J  {\frac{\partial \mathbf L}{\partial \pmb \eta}} = \tilde{\frac{\partial \mathbf q}{\partial \pmb \eta}} J  {\frac{\partial}{\partial \pmb \eta}}\left(\mathbf q \times \mathbf p\right) \\
            &= \tilde{\frac{\partial \mathbf q}{\partial \pmb \eta}} J \left[\frac{\partial \mathbf p}{\partial \pmb \eta} \left( \mathbf q \cdot \bar{\bar{\pmb \epsilon}}
            \right) - \frac{\partial \mathbf q}{\partial \pmb \eta} \left( \mathbf p\cdot \bar{\bar{\pmb \epsilon}}
            \right)\right]\\
            &= \tilde{\frac{\partial \mathbf q}{\partial \pmb \eta}} J \frac{\partial \mathbf p}{\partial \pmb \eta} \left( \mathbf q \cdot \bar{\bar{\pmb \epsilon}}
            \right) - \tilde{\frac{\partial \mathbf q}{\partial \pmb \eta}} J \frac{\partial \mathbf q}{\partial \pmb \eta} \left( \mathbf p\cdot \bar{\bar{\pmb \epsilon}}
            \right)\\
            &= -\cancelto{0}{\left\{ \mathbf q, \mathbf q \right\}} \left( \mathbf p \cdot \bar{\bar{\pmb \epsilon}}
            \right) + \left\{ \mathbf q, \mathbf p \right\} \left( \mathbf q\cdot \bar{\bar{\pmb \epsilon}}
            \right)\\
            &=  \cancelto{\mathbb{I}}{\left\{ \mathbf q, \mathbf p \right\}} \left( \mathbf q\cdot \bar{\bar{\pmb \epsilon}}\right) = \left( \mathbf q\cdot \bar{\bar{\pmb \epsilon}}\right)\\
        \end{align*}

        \begin{align*}
            \left\{\mathbf{L} , \mathbf{L}\right\} &=  \tilde{\frac{\partial \mathbf L}{\partial \pmb \eta}} J  {\frac{\partial \mathbf L}{\partial \pmb \eta}} = \tilde{\frac{\partial \mathbf L}{\partial \pmb \eta}} J  {\frac{\partial}{\partial \pmb \eta}}\left(\mathbf q \times \mathbf p\right) \\
            &= \tilde{\frac{\partial \mathbf L}{\partial \pmb \eta}} J \left[\frac{\partial \mathbf p}{\partial \pmb \eta} \left( \mathbf q \cdot \bar{\bar{\pmb \epsilon}}
            \right) - \frac{\partial \mathbf q}{\partial \pmb \eta} \left( \mathbf p\cdot \bar{\bar{\pmb \epsilon}}
            \right)\right]\\
            &= \tilde{\frac{\partial \mathbf L}{\partial \pmb \eta}} J \frac{\partial \mathbf p}{\partial \pmb \eta} \left( \mathbf q \cdot \bar{\bar{\pmb \epsilon}}
            \right) - \tilde{\frac{\partial \mathbf L}{\partial \pmb \eta}} J \frac{\partial \mathbf q}{\partial \pmb \eta} \left( \mathbf p\cdot \bar{\bar{\pmb \epsilon}}
            \right)\\
            &= \left\{ \mathbf L, \mathbf p \right\} \left( \mathbf q \cdot \bar{\bar{\pmb \epsilon}}
            \right) - \left\{ \mathbf L, \mathbf q \right\} \left( \mathbf p\cdot \bar{\bar{\pmb \epsilon}}
            \right)\\
            &= -\left( \mathbf p \cdot \bar{\bar{\pmb \epsilon}}\right) \left( \mathbf q \cdot \bar{\bar{\pmb \epsilon}}
            \right) + \left( \mathbf q \cdot \bar{\bar{\pmb \epsilon}}\right) \left( \mathbf p\cdot \bar{\bar{\pmb \epsilon}}
            \right)\\
            &=  \left( \mathbf q \mathbf p - \mathbf p \mathbf q\right) \cdot \bar{\bar{\pmb \epsilon}} \\
            &=  \mathbf{L} \cdot \bar{\bar{\pmb \epsilon}} \\
        \end{align*}
        Conclusion: Todos aquellos tensores de rango 2 tales que: 
        \begin{align}
            \mathbf T = \mathbf t \cdot \bar{\bar{\pmb \epsilon}}  
        \end{align}
        para algun vector $\mathbf t$ son pseudotensores antisimetricos y ¿son invariantes bajo transformaciones canonicas?.

        ¿Es valida el resultado (23.2) para cualquier tensores de rango mayor que 1? 
        permitiendo de esta forma definir el producto cruz para tensores de rango superior a 1, utilizando el tensor de Levi-Civita de rango correspondiente

    \end{answer}





    24. Resolver el problema del movimiento de un proyectil puntiforme en un plano vertical, utilizando el método de Hamilton-Jacobi. Hallar la ecuación de la trayectoria y la dependencia del tiempo de las coordenadas, suponiendo que el proyectil se ha disparado en el instante $t=0$ desde el origen con una velocidad $v_0$ que forma un ángulo $\alpha$ con la horizontal.

    \begin{answer}[Punto 10]
        La entropia esta definida por:
        \begin{align*}
            dQ = TdS \quad &\Rightarrow \quad \left( \frac{dQ}{dT}\right)_{\mathscr{H}} = T \left(\frac{dS}{dT}\right)_\mathscr{H}\\
            &\Rightarrow \quad C_{\mathscr{H}} = T \left(\frac{dS}{dT}\right)_{\mathscr{H}}\\
            &\Rightarrow \quad dS = \frac{C_{\mathscr{H}}}{T}dT\\
            &\Rightarrow \quad dS = \left(\frac{B+C \mathscr{H}_0^2}{T^2}+D T^2\right)\frac{dT}{T}\\
            &\Rightarrow \quad S = \int_{T_i}^{T_f} \left(\frac{B+C \mathscr{H}_0^2}{T^2}+D T^2\right)\frac{dT}{T}\\
            &\Rightarrow \quad S = \int_{T_i}^{T_f} \left(\frac{B+C \mathscr{H}_0^2}{T^3}+D T\right)dT\\
            &\Rightarrow \quad S = \left[-\frac{B+C \mathscr{H}_0^2}{2T^2}+D \frac{T^2}{2}\right]_{T_i}^{T_f}\\
            &\Rightarrow \quad S = -\frac{B+C \mathscr{H}_0^2}{2T_f^2}+D \frac{T_f^2}{2} + \frac{B+C \mathscr{H}_0^2}{2T_i^2}-D \frac{T_i^2}{2}\\
            &\Rightarrow \quad S = \frac{B+C \mathscr{H}_0^2}{2}\left(\frac{1}{T_i^2}-\frac{1}{T_f^2}\right)+D \left(\frac{T_f^2}{2} -\frac{T_i^2}{2}\right)\\
            &\Rightarrow \quad S = \frac{B+C \mathscr{H}_0^2}{2}\left(\frac{T_f^2 - T_i^2}{T_i^2T_f^2}\right)+D \left(\frac{T_f^2 - T_i^2}{2}\right)\\
            &\Rightarrow \quad S =\frac{T_f^2 - T_i^2}2\left[ \left(\frac{B+C \mathscr{H}_0^2}{T_i^2T_f^2}\right)+D \left(T_f^2 - T_i^2\right)\right]\\
        \end{align*}
    \end{answer}
\end{document}