\documentclass[a4paper]{article}
\usepackage{student}
\usepackage{graphicx}

\usepackage{cancel}

% Metadata
\date{\today}

%-------------------------------%
% Other details
% TODO: Fill these
%-------------------------------%
\title{Ejercicio 9}
\setmembername{Cristian Camilo Pérez Puentes} 

%-------------------------------%
% Add / Delete commands and packages
% TODO: Add / Delete here as you need
%-------------------------------%
\usepackage{amsmath,amssymb,bm}
\usepackage[spanish]{babel}

% Custom your usual commands here. Renew these.
\newcommand{\KL}{\mathrm{KL}}
\newcommand{\R}{\mathbb{R}}
\newcommand{\E}{\mathbb{E}}
\newcommand{\T}{\top}
\newcommand{\expdist}[2]{%
        \normalfont{\textsc{Exp}}(#1, #2)%
    }
\newcommand{\expparam}{\bm \lambda}
\newcommand{\Expparam}{\bm \Lambda}
\newcommand{\natparam}{\bm \eta}
\newcommand{\Natparam}{\bm H}
\newcommand{\sufstat}{\bm u}

% Main document
\begin{document}
    % Add header
    \header{}

    % Use `answer` environment to add solutions
    % \begin{answer}[Question 1.1] for example starts an environment formatted
    % for Question 1.1


    A box containing a particle is divided into a right and a left compartment by a thin
    partition. If the particle is known to be on the right (left) side with certainty, the state
    is represented by the position eigenket $|R\rangle (|L\rangle)$, where we have neglected spatial
    variations within each half of the box. The most general state vector can then be
    written as
\begin{equation}
    |\alpha\rangle = |R\rangle\langle R|\alpha \rangle+|L\rangle\langle L|\alpha\rangle
\end{equation}
where $\langle R|\alpha\rangle$ and  $\langle L|\alpha \rangle$ can be regarded as “wave functions.” The particle can tunnel
through the partition; this tunneling effect is characterized by the Hamiltonian
\begin{equation}
    H = \Delta(|L\rangle \langle R|+|R\rangle \langle L|),
\end{equation}
where $\Delta$ is a real number with the dimension of energy.
\begin{itemize}
    \item [a.] Find the normalized energy eigenkets. What are the corresponding energy 
    eigenvalues?
    \item [b.] In the Schrodinger picture the base kets $|R\rangle$ and $|L\rangle$ 
    are fixed, and the state vector moves with time. Suppose the system is represented 
    by $|\alpha \rangle$ as given above at $t = 0$. Find the state vector 
    $|\alpha,t_0 = 0; t\rangle $ for $t > 0$ by applying the appropriate 
    time-evolution operator to $|\alpha\rangle$.
    \item [c.] Suppose at $t = 0$ the particle is on the right side with certainty. What is the
    probability for observing the particle on the left side as a function of time?
    \item [d.] Write down the coupled Schrodinger equations for the wave functions
    $R|\alpha,t_0 = 0; t\rangle$ and $L|\alpha,t_0 = 0; t\rangle$. Show that the 
    solutions to the coupled Schrodinger equations are just what you expect from (b). 
    \item [e.] Suppose the printer made an error and wrote $H$ as 
    $H = \Delta|L\rangle \langle R|$. 

    By explicitly solving the most general time-evolution problem with this Hamiltonian, show that probability conservation is violated.
\end{itemize}
    \begin{answer}
        \begin{itemize}
            \item [a.]Los autoestados de energia son aquellos que:
            $$H|h'\rangle = E_{h'}|h'\rangle \qquad  \Rightarrow $$
            \begin{align*}
                H|h'\rangle - E_{h'}|h'\rangle &= 0 \\
                \Delta(|L\rangle \langle R|+|R\rangle \langle L|)|h'\rangle - E_{h'}|h'\rangle &= 0 \\
                \Delta|L\rangle \langle R|h'\rangle + \Delta|R\rangle \langle L|h'\rangle - E_{h'}|h'\rangle &= 0 \\
                \Delta\langle L |L\rangle \langle R|h'\rangle + \Delta\langle L |R\rangle \langle L|h'\rangle - E_{h'}\langle L |h'\rangle &= 0 \\
                \Delta \langle R|h'\rangle - E_{h'}\langle L |h'\rangle &= 0 \\
            \end{align*}
        ademas 
        \begin{align*}
            \Delta|L\rangle \langle R|h'\rangle + \Delta|R\rangle \langle L|h'\rangle - E_{h'}|h'\rangle &= 0 \\
            \Delta\langle R |L\rangle \langle R|h'\rangle + \Delta\langle R |R\rangle \langle L|h'\rangle - E_{h'}\langle R|h'\rangle &= 0 \\
            \Delta \langle L|h'\rangle - E_{h'}\langle R |h'\rangle &= 0 \\
        \end{align*}
        por lo que
        \begin{align*}(\Delta - E_{h'}) \langle R |h'\rangle + (\Delta - E_{h'})\langle L |h'\rangle &= 0\\
            &= ( \Delta - E_{h'}) (\langle R |h'\rangle + \langle L |h'\rangle) = 0\\
        \end{align*}
        \begin{align*}(\Delta + E_{h'}) \langle R |h'\rangle - (\Delta + E_{h'})\langle L |h'\rangle &= 0\\
            &= ( \Delta + E_{h'}) (\langle R |h'\rangle + \langle L |h'\rangle) = 0\\
        \end{align*}
        Y como $\langle R |h'\rangle$ y $\langle L |h'\rangle$ son funciones de onda que en general no van a coincidir en todos los valores 
        de $x$ y $t$, por lo que 
        $$E_{h'} = \pm \Delta$$
        son los autolvalores de energia. Usando estos autolvalores en la ecuacion de autolvalores para $H$ entonces
        \begin{align*}
            H|h'\rangle - E_{h'}|h'\rangle &= 0 \\
            \Delta(|L\rangle \langle R|+|R\rangle \langle L|)|h'\rangle - \Delta|h'\rangle &= 0 \\
            \Delta|L\rangle \langle R|h'\rangle + \Delta|R\rangle \langle L|h'\rangle - \Delta|h'\rangle &= 0 \\
            \left(
                |L\rangle \langle R|+|R\rangle \langle L| - 1
            \right) \Delta|h'\rangle &= 0 \\
            \left(
                |L\rangle \langle R|+|R\rangle \langle L| - |R\rangle \langle R| - |L\rangle \langle L|
            \right) \Delta |h'\rangle &= 0 \\
            |L\rangle \langle R|h'\rangle +|R\rangle \langle L|h'\rangle  - |R\rangle \langle R|h'\rangle  - |L\rangle \langle L| h'\rangle &= 0 \\
            \langle R|h'\rangle - \langle L| h'\rangle &= 0 \\
            \langle L|h'\rangle - \langle R| h'\rangle &= 0 \\
        \end{align*}
        Si bien este ultimo procedimiento no es necesario en este ejercicio para encontra los autoestados de energia, pues es suficiente con
        reemplazar los autovalores de energia ya sea alguna de las ecuaciones $\Delta \langle R|h'\rangle - E_{h'}\langle L |h'\rangle = 0$ o
        $\Delta \langle L|h'\rangle - E_{h'}\langle R |h'\rangle = 0$ e inferir $|h'\rangle$. Este procedimiento muestra un analogo a resolver
        la equacion de autolvalores, sin usar el equivalente de los operadores y estados en matrices, en donde el equivaltente a la ecuacion en
        representacion matricial:
        $$ (\mathbb H - \mathbb I \Delta) \mathbf h' = 0$$
        es 
        $$
        \left(|L\rangle \langle R|+|R\rangle \langle L| - 1
        \right) \Delta|h'\rangle = \left((|L\rangle \langle R|+|R\rangle \langle L)\Delta - 1\Delta
        \right) |h'\rangle = 0$$
        donde
        $$\Delta(|L\rangle \langle R|+|R\rangle \langle L|) \doteq \mathbb{H} \qquad |h'\rangle \doteq \mathbf h'$$
        $$1\Delta = \Delta\sum_{b'} \Lambda_{b'} = \Delta(|L\rangle \langle L| + |R\rangle \langle R|) \doteq \mathbb{I}\Delta$$
        y como los coeficientes $c_L = \langle L| h'\rangle$ y $c_R = \langle R| h'\rangle$ pueden representar las proyecciones en algun 
        instante $x_0,t_0$ del autoestados $|h'\rangle$ en los estados $|L\rangle$ y $|R\rangle$ respectivamente, entonces para ese 
        instante la relacion 
    
        $$c_L = c_R \qquad \Leftrightarrow \qquad \langle R|h'\rangle = \langle L| h'\rangle =h' $$
        Es el equivalente a al encontrar un relacion entre la componentes $\mathbf h'$ tipo $h_1 = h_2$ al solucionar la representacion 
        matricial de la ecuacion de autolvalores para $\mathbb H$, por algun metodo de solucion de sistemas de ecuaciones lineales, 
        la cual tiene una infinitas soluciones no triviales, escogiendose alguna solucion que de una representacion normalizada de los 
        autoestados de $\mathbb H$.
    
        De lo anterior dicho,  $|h'\rangle = h'|L \rangle + h'|R\rangle$ de la misma forma en la que $\mathbf h'$ se puede descompener 
        en la superposicion $|L\rangle \doteq \mathbf{e_1} $ y $|R\rangle \doteq \mathbf{e_2}$. Lo que se puede verificar, pues
        \begin{align*}
            \langle R|h'\rangle - \langle L| h'\rangle &= \\
            \langle R|h'(|L \rangle + |R\rangle) - \langle L|h'(|L \rangle + |R\rangle) &= 0 \\
            h\langle R|L \rangle +h \langle R|R\rangle - h\langle L| L\rangle - h\langle L|R\rangle &= 0 \\
            0 + h - h - 0 &= 0 \\
        \end{align*}
        Y de la misma forma reemplazando $E_h = -\Delta$ en $\Delta \langle R|h'\rangle - E_{h'}\langle L |h'\rangle = 0$ se obtiene
        $$-h'' = \langle R|h''\rangle = - \langle L| h''\rangle$$
        por lo que $|h''\rangle = h''|L \rangle - h''|R\rangle$.
        Normalizando los autoestados de energia
        \begin{align*}
            \langle h'|h'\rangle &= (h'\langle L| + h'\langle R|)(h'|L \rangle + h'|R\rangle) \\
            &= h'^2 + h'^2 = 2h'^2 = 1 \qquad \Rightarrow \qquad \\
            h' &= \frac{1}{\sqrt{2}} \qquad \Rightarrow \qquad \\
            |h'\rangle&= \frac{1}{\sqrt{2}}(|L \rangle + |R\rangle) 
        \end{align*}
        \begin{align*}
            \langle h''|h''\rangle &= (h''\langle L| - h''\langle R|)(h''|L \rangle - h''|R\rangle) \\
            &= h''^2 + h''^2 = 2h''^2 = 1 \qquad \Rightarrow \qquad \\
            h'' &= \frac{1}{\sqrt{2}} \qquad \Rightarrow \qquad \\
            |h''\rangle&= \frac{1}{\sqrt{2}}(|L \rangle - |R\rangle)
        \end{align*}    
        Luego los autoestados de energia normalizados son $|h'\rangle = \frac{1}{\sqrt{2}}(|L \rangle + |R\rangle)$ y $|h''\rangle = \frac{1}{\sqrt{2}}(|L \rangle - |R\rangle)$, 
        con correspondientes autovalores de energia $E_{h'} = \Delta$ y $E_{h''} = -\Delta$ respectivamente.

        \item[b.] Como los kets base son fijos y el Hamiltoniano esta representado en terminos de los kets base y una constante real $\Delta$
        entonces el Hamiltoniano es constante en el tiempo y por lo que el poderador evolucion temporal tiene la forma
        $$U(t) = \exp{\left[-\frac{iHt}{\hbar}\right]}$$  
        El cual expandido en terminos de eigenkets de $H$ base es:
        \begin{align*}
            U(t) &= |h'\rangle \exp{\left[-\frac{i\Delta}{\hbar}\right]} \langle h'| + |h''\rangle \exp{\left[\frac{i\Delta t}{\hbar}\right]} \langle h''|\\
            &= \frac{1}{\sqrt{2}}(|L \rangle + |R\rangle) \exp{\left[-\frac{i\Delta t}{\hbar}\right]} \frac{1}{\sqrt{2}}(\langle L| + \langle R|)+ \frac{1}{\sqrt{2}}(|L \rangle - |R\rangle) \exp{\left[\frac{i\Delta t}{\hbar}\right]} \frac{1}{\sqrt{2}}(\langle L| - \langle R|)\\
        \end{align*} 
    \end{itemize}
    por lo que la evolucion temporal del operador de estado $|\alpha \rangle = |\alpha,t_0=0\rangle = |L\rangle \langle L|\alpha\rangle + |R\rangle \langle R|\alpha\rangle$ es 

    \begin{align*}
        |\alpha,t\rangle &= U(t)|\alpha\rangle \\
        &= \frac{1}{\sqrt{2}}(|L \rangle + |R\rangle) \exp{\left[-\frac{i\Delta t}{\hbar}\right]} \frac{1}{\sqrt{2}}(\langle L| + \langle R|)|\alpha\rangle + \frac{1}{\sqrt{2}}(|L \rangle - |R\rangle) \exp{\left[\frac{i\Delta t}{\hbar}\right]} \frac{1}{\sqrt{2}}(\langle L| - \langle R|)|\alpha\rangle\\
        &= \frac{1}{2}(|L \rangle + |R\rangle) \exp{\left[-\frac{i\Delta t}{\hbar}\right]} (\langle L|\alpha\rangle + \langle R|\alpha\rangle) + \frac{1}{2}(|L \rangle - |R\rangle) \exp{\left[\frac{i\Delta t}{\hbar}\right]} (\langle L|\alpha\rangle - \langle R|\alpha\rangle)\\
        &= \frac{1}{2}(|L \rangle + |R\rangle) \exp{\left[-\frac{i\Delta t}{\hbar}\right]} (\langle L|(|L\rangle \langle L|\alpha\rangle + |R\rangle \langle R|\alpha\rangle) + \langle R|(|L\rangle \langle L|\alpha\rangle + |R\rangle \langle R|\alpha\rangle)) \\
        &+ \frac{1}{2}(|L \rangle - |R\rangle) \exp{\left[\frac{i\Delta t}{\hbar}\right]} (\langle L|(|L\rangle \langle L|\alpha\rangle + |R\rangle \langle R|\alpha\rangle) - \langle R|(|L\rangle \langle L|\alpha\rangle + |R\rangle \langle R|\alpha\rangle))\\
        &= \frac{1}{2}(|L \rangle + |R\rangle) \exp{\left[-\frac{i\Delta t}{\hbar}\right]} (\langle L|\alpha\rangle + \langle R|\alpha\rangle) + \frac{1}{2}(|L \rangle - |R\rangle) \exp{\left[\frac{i\Delta t}{\hbar}\right]} (\langle L|\alpha\rangle - \langle R|\alpha\rangle)\\
        &= \frac{1}{2}(|L \rangle \langle L | \alpha \rangle + |L \rangle \langle R | \alpha \rangle + |R \rangle \langle L | \alpha \rangle + |R \rangle \langle R | \alpha \rangle) \exp{\left[-\frac{i\Delta t}{\hbar}\right]} \\
        &+ \frac{1}{2}(|L \rangle \langle L | \alpha \rangle - |L \rangle \langle R | \alpha \rangle - |R \rangle \langle L | \alpha \rangle + |R \rangle \langle R | \alpha \rangle) \exp{\left[\frac{i\Delta t}{\hbar}\right]} \\
    \end{align*}
    \end{answer}
\end{document}