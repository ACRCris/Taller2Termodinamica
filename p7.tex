7. Un mol de un sistema paramagnético ideal obedece la ley de Curie.
$$
\mathscr{M} =\frac{C_{C} \mathscr H}{T}
$$
Donde $\mathscr M$ es la magnetización y $\mathscr H$ es un campo magnético externo, con constante de Curie $C_C$. Suponga que la energía interna $U$ es función de $T$ únicamente, de modo que $\mathrm{dU}=\mathrm{C}_{\mathrm{V},\mathscr M}  \mathrm{dT}$, donde $\mathrm{C}_{\mathrm{v}, \mathscr{M}}$ es una capacidad calorífica a volumen y magnetización constantes. Demostrar que la ecuación de la familia de superficies adiabáticas es
$$
\frac{C_{V, \mathscr M}}{n R} \ln T+\ln V=\frac{\mu_0 \mathscr M^2}{2 n R C_C}+\ln A,
$$
Donde A es una constante.